\documentclass[a4paper]{article}
%\usepackage[spanish]{babel}
\usepackage[utf8]{inputenc}
\usepackage{amsfonts}
\usepackage{amsmath}
\usepackage{graphicx}
\usepackage{float}
%\graphicspath{ {images/} }
\usepackage{hyperref}
\usepackage{enumerate}
\title {\fbox{\Huge{\textbf{Ejercicio Tema 4}}}}
\author {\fbox{Ana Buendía Ruiz-Azuaga}}
%\date {}
\begin{document}
\maketitle
%\tableofcontents


\section{Ejercicio Tema 4}

\textbf{Los parámetros de un criptosistema de ElGamal son p = 211 y g = 3, es decir, el criptosistema está diseñado en el cuerpo $F_{211}$ = $Z_{211}$ y tomamos como generador de $F^*_{211}$ , g = 3. La clave pública empleada es $3^a = 109 \mod 211$. Descifra el criptograma (154, dni mod 211), donde dni es el número de tu DNI. Para calcular los logaritmos discretos necesarios emplea dos de los métodos descritos en la teoría.}

Comenzamos calculando $a$. Para ello vamos a usar dos implementaciones distintas del logaritmo discreto: paso de bebé - paso de gigante y el algoritmo de Silver-Pohlig-Hellman.

Una vez calculado $a$, descifraremos el criptograma.

\subsection{Paso de bebé - paso de gigante}

Sea $p=211$ y $g=3$. Vamos a calcular $a=\log_3 (109)$

Comenzamos calculando f como $f=\lceil\sqrt{p-1}\rceil=\lceil\sqrt{210}\rceil=15$.

Ahora calculamos la tabla, obteniendo:

\begin{verbatim}
[0, 1], [1, 3], [2, 9], [3, 27], [4, 81], [5, 32], [6, 96], [7, 77], 
[8, 20], [9, 60], [10, 180], [11, 118], [12, 143], [13, 7], [14, 21]
\end{verbatim}

También calculamos $g^{-f}=g^{p-1-f}=67 \mod p$.

Empezando por $h_0=109$, vamos a calcular parejas $(i,h_i)$, con $h_i=h_{i-1}g^{-f}\mod p$ hasta que alguno de estos $h_i$ pertenezca a la tabla ya calculada.

Las parejas calculadas son: $(0,109)$, $(1,129)$, $(2,203)$, $(3,97)$, $(4,169)$, $(5,140)$, $(6,96)$.

Obtenemos la pareja $(6,96)$, donde $96$ está en la tabla en la posición $6$.

Luego tenemos $a=j+if=6+6\cdot 15=96$.

\subsection{Algoritmo de Silver-Pohlig-Hellman}

Partimos del mismo planteamiento del apartado anterior: $p=211$ y $g=3$. Vamos a calcular $a=\log_3 (109)$.

Consideramos $n=p-1=210$.

Tenemos que la descomposición de $n=210=2 \cdot 3 \cdot 5\cdot 7$.

Comenzamos trabajando con $p_1=2$, $e_1=1$. Tenemos que:

$r_{1,0}=1$, $r_{1,1}=210$.

Y la sucesión de coeficientes:

$y_0 = 109$, $x_0 = 0$.

Calculamos además $y_0^{\frac{n}{p_1^{0+1}}}=109^{\frac{210}{2}}\equiv 1\mod p$.

Luego $m_1=0$.

En el siguiente paso, con $p_2=3$, $e_2=1$:

$r_{2,0}=1$, $r_{2,1}=196$, $r_{2,2}=14$.

Y la sucesión de coeficientes:

$y_0 = 109$, $x_0 = 0$.

Calculamos además $y_0^{\frac{n}{p_2^{0+1}}}=109^{\frac{210}{3}}\equiv 1\mod p$.


Luego $m_2=0$.

Ahora con $p_3=5$, $e_3=1$:

$r_{3,0}=1$, $r_{3,1}=188$, $r_{3,2}=107$, $r_{3,3}=71$, $r_{3,4}=55$.

Y la sucesión de coeficientes:

$y_0 = 109$, $x_0 = 1$.

Calculamos además $y_0^{\frac{n}{p_3^{0+1}}}=109^{\frac{210}{5}}\equiv 188\mod p$.


Luego $m_3=1$.

Finalmente con $p_4=7$, $e_4=1$:

$r_{4,0}=1$, $r_{4,1}=171$, $r_{4,2}=123$, $r_{4,3}=144$, $r_{4,4}=148$, $r_{4,5}=199$, $r_{4,6}=58$.

Y la sucesión de coeficientes:

$y_0 = 109$, $x_0 = 5$.

Calculamos además $y_0^{\frac{n}{p_4^{0+1}}}=109^{\frac{210}{7}}\equiv 199\mod p$.


Luego $m_4=5$.

Por tanto el sistema de congruencias es:

\begin{equation}
\begin{aligned}
m & \equiv 0 \mod 2 \\
m & \equiv 0 \mod 3 \\
m & \equiv 1 \mod 5 \\
m & \equiv 5 \mod 7 \\
\end{aligned}
\end{equation}

Resolvemos este sistema de congruencias con ayuda de software y obtenemos que $m=96$.

Dado que la solución obtenida por los dos algoritmos es la misma, procedemos a descifrar el criptograma.

\subsection{Descifrando el criptograma}

Mi DNI es 77770080, y $77770080 \mod 211 \equiv 122$, luego el criptograma que vamos a descifrar es $(154,122)$.

Tenemos que $D_{a}(x,y)=yx^{-a}$, sustituyento $D_{96}(154,122)=122 \cdot 154^{-96}=193$.

Luego el mensaje original es $193$.




\end{document}