\documentclass[a4paper]{article}
%\usepackage[spanish]{babel}
\usepackage[utf8]{inputenc}
\usepackage{amsfonts}
\usepackage{amsmath}
\usepackage{graphicx}
\usepackage{float}
%\graphicspath{ {images/} }
\usepackage{hyperref}
\usepackage{enumerate}
\title {\fbox{\Huge{\textbf{Ejercicio Tema 4}}}}
\author {\fbox{Ana Buendía Ruiz-Azuaga}}
%\date {}
\begin{document}
\maketitle
%\tableofcontents


\section{Ejercicio Tema 4}

\textbf{**Los parámetros de un criptosistema de ElGamal son p = 211 y g = 3, es decir, el criptosistema está diseñado en el cuerpo $F_{211}$ = $Z_{211}$ y tomamos como generador de $F^*_{211}$ , g = 3. La clave pública empleada es $3^a = 109 \\mod 211$. Descifra el criptograma (154, dni mod 211), donde dni es el número de tu DNI. Para calcular los logaritmos discretos necesarios emplea dos de los métodos descritos en la teoría.**}

Comenzamos calculando $a$. Para ello vamos a usar dos implementaciones distintas del logaritmo discreto: paso de bebé - paso de gigante y el algoritmo de Silver-Pohlig-Hellman.

Una vez calculado $a$, descifraremos el criptograma.

\subsection{Paso de bebé - paso de gigante}



\subsection{Algoritmo de Silver-Pohlig-Hellman}



\subsection{Descifrando el criptograma}






\end{document}