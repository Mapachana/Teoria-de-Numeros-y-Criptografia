\documentclass[a4paper]{article}
%\usepackage[spanish]{babel}
\usepackage[utf8]{inputenc}
\usepackage{amsfonts}
\usepackage{amsmath}
\usepackage{graphicx}
\usepackage{float}
%\graphicspath{ {images/} }
\usepackage{hyperref}
\usepackage{enumerate}
\title {\fbox{\Huge{\textbf{Ejercicio Tema 2}}}}
\author {\fbox{Ana Buendía Ruiz-Azuaga}}
%\date {}
\begin{document}
\maketitle
%\tableofcontents


\section{Ejercicio Tema 2}

\textbf{Consideremos el cifrado por bloques MiniAES descrito en
el ejercicio 2.1.}

\subsection{Apartado 1}

\textbf{Calcula $\text{E}_{dni}$(0x01234567) usando el modo CBC e IV = 0x0001.}

Trabajamos en $\mathbb{F}_{16}=\mathbb{F}_2(\xi )_{\xi^4+\xi+1}$.

Del ejercicio 2.1, vamos a usar la definición explícita de $\gamma$ que calculamos en clase, y además consideramos las funciones que, dado

$$\begin{pmatrix} a_0 & a_2 \\ a_1 & a_3 \end{pmatrix} \in \mathbb{F}_{16}^{2\times 2}$$

se definen:

$$\gamma \begin{pmatrix} a_0 & a_2 \\ a_1 & a_3 \end{pmatrix} = \begin{pmatrix} \gamma (a_0) & \gamma (a_2) \\ \gamma (a_1) & \gamma (a_3) \end{pmatrix}$$

$$\pi \begin{pmatrix} a_0 & a_2 \\ a_1 & a_3 \end{pmatrix} = \begin{pmatrix} a_0 & a_2 \\ a_3 & a_1 \end{pmatrix}$$

$$\theta \begin{pmatrix} a_0 & a_2 \\ a_1 & a_3 \end{pmatrix} = \begin{pmatrix} 0011 & 0010 \\ 0010 & 0011 \end{pmatrix}\begin{pmatrix} a_0 & a_2 \\ a_1 & a_3 \end{pmatrix}$$

$$\sigma_{K_i} \begin{pmatrix} a_0 & a_2 \\ a_1 & a_3 \end{pmatrix} = \begin{pmatrix} a_0 & a_2 \\ a_1 & a_3 \end{pmatrix} + \begin{pmatrix} k_{i,0} & k_{i,2} \\ k_{i,1} & k_{i,3} \end{pmatrix}$$



Tenemos que $k=k_0k_1k_2k_3$ y se define $E_K=\sigma_{K_2} \circ \pi \circ \gamma \circ \sigma_{K_1} \circ \theta \circ \pi \circ \gamma \circ \sigma_{K_0}$.

Ahora calculamos nuestro dni módulo 65536, resultando $77770080\mod 65536\equiv 44384$.

Y, pasanod 44384 a binario y hexadecimal tenemos que:

$$k = 1010110101100000 = 0xAD60.$$

Además, sabemos que:

$$c_{[0]}=0x0001=0000 0000 0000 0001$$

que tiene 16 dígitos.

Ahora, dividimos el mensaje en bloques de esta longitud:

$$m = 0x01234567 = 0000 0001 0010 0011 0100 0101 0110 0111$$

Resultando así dos bloques, $m_1=0000 0001 0010 0011$ y $m_2=0100 0101 0110 0111$, luego tenemos que l=2.

A continuación calculamos:

Calculamos $K_0$:

$$w_0 = k_0 = A = 1010$$
$$w_1 = k_1 = D = 1101$$
$$w_2 = k_2 = 6 = 0110$$
$$w_3 = k_3 = 0 = 0000$$

Calculamos $K_1$:

$$w_4 = w_0 \oplus \gamma (w_3) \oplus 0001 = 1010 \oplus \gamma (0000) \oplus 0001 = 1000$$
$$w_5 = w_1 \oplus w_4 = 1101 \oplus 1000 = 0101$$
$$w_6 = w_2 \oplus w_5 = 0110 \oplus 0101 = 0011$$
$$w_7 = w3_ \oplus w_6 = 0000 \oplus 0011 = 0011$$

Calculamos $K_2$:

$$w_8 = w_4 \oplus \gamma (w_7) \oplus 0010 = 1000 \oplus \gamma (0011) \oplus 0010 = 1101$$
$$w_9 = w_5 \oplus w_8 = 0101 \oplus 1101 = 1000$$
$$w_{10} = w_6 \oplus w_9 = 0011 \oplus 1000 = 1011$$
$$w_{11} = w_7 \oplus w_{10} = 0011 \oplus  1011 = 1000$$

Comenzamos calculando $E_K (m_1 \oplus c_{[0]} ) = E_K ( 0000 0001 0010 0011 \oplus 0000 0000 0000 0001) = E_K(0000 0001 0010 0010)$.

Aplicamos en orden por tanto:

$$\sigma_{K_0} \begin{pmatrix} 0000 & 0010 \\ 0001 & 0010 \end{pmatrix} = \begin{pmatrix} 0000 & 0010 \\ 0001 & 0010 \end{pmatrix} + \begin{pmatrix} 1010 & 0110 \\ 1101 & 0000 \end{pmatrix} = \begin{pmatrix} 1010 & 0100 \\ 1100 & 0010 \end{pmatrix}$$

$$\gamma \begin{pmatrix} 1010 & 0100 \\ 1100 & 0010 \end{pmatrix} = \begin{pmatrix} 1010 & 0001 \\ 1001 & 1111 \end{pmatrix}$$

$$\pi \begin{pmatrix} 1010 & 0001 \\ 1001 & 1111 \end{pmatrix} = \begin{pmatrix} 1010 & 0001 \\ 1111 & 1001 \end{pmatrix}$$

$$\theta \begin{pmatrix} 1010 & 0001 \\ 1111 & 1001 \end{pmatrix} = \begin{pmatrix} 0011 & 0010 \\ 0010 & 0011 \end{pmatrix}\begin{pmatrix} 1010 & 0001 \\ 1111 & 1001 \end{pmatrix} = \begin{pmatrix} 0000 & 0010 \\ 0101 & 1010 \end{pmatrix}$$

$$\sigma_{K_1} \begin{pmatrix} 0000 & 0010 \\ 0101 & 1010 \end{pmatrix} = \begin{pmatrix} 0000 & 0010 \\ 0101 & 1010 \end{pmatrix} + \begin{pmatrix} 1000 & 0011 \\ 0101 & 0011 \end{pmatrix} = \begin{pmatrix} 1000 & 0001 \\ 0000 & 1001 \end{pmatrix}$$

$$\gamma \begin{pmatrix} 1000 & 0001 \\ 0000 & 1001 \end{pmatrix} = \begin{pmatrix} 1100 & 1000 \\ 0011 & 1110 \end{pmatrix}$$

$$\pi \begin{pmatrix} 1100 & 1000 \\ 0011 & 1110 \end{pmatrix} = \begin{pmatrix} 1100 & 1000 \\ 1110 & 0011 \end{pmatrix}$$

$$\sigma_{K_2} \begin{pmatrix} 1100 & 1000 \\ 1110 & 0011 \end{pmatrix} = \begin{pmatrix} 1100 & 1000 \\ 1110 & 0011 \end{pmatrix} + \begin{pmatrix} 1101 & 1011 \\ 1000 & 1000 \end{pmatrix} = \begin{pmatrix} 0001 & 0011 \\ 0110 & 1011 \end{pmatrix}$$

luego $c_{[1]} = 0001 0110 0011 1011$ y pasamos a calcular $c_{[2]} = E_K(m_2 \oplus c_{[1]}$.

$$E_K(m_2 \oplus c_{[1]}) = E_K( 0100 0101 0110 0111 \oplus 0001 0110 0011 1011) = E_K(0101 0011 0101 1100).$$

Aplicamos:

$$\sigma_{K_0} \begin{pmatrix} 0101 & 0101 \\ 0011 & 1100 \end{pmatrix} = \begin{pmatrix} 0101 & 0101 \\ 0011 & 1100 \end{pmatrix} + \begin{pmatrix} 1010 & 0110 \\ 1101 & 0000 \end{pmatrix} = \begin{pmatrix} 1111 & 0011 \\ 1110 & 1100 \end{pmatrix}$$

$$\gamma \begin{pmatrix} 1111 & 0011 \\ 1110 & 1100 \end{pmatrix} = \begin{pmatrix} 0100 & 0111 \\ 0101 & 1001 \end{pmatrix}$$

$$\pi \begin{pmatrix} 0100 & 0111 \\ 1110 & 1001 \end{pmatrix} = \begin{pmatrix} 0100 & 0111 \\ 1001 & 0101 \end{pmatrix}$$

$$\theta \begin{pmatrix} 0100 & 0111 \\ 1001 & 0101 \end{pmatrix} = \begin{pmatrix} 0011 & 0010 \\ 0010 & 0011 \end{pmatrix}\begin{pmatrix} 0100 & 0111 \\ 1001 & 0101 \end{pmatrix} = \begin{pmatrix} 1101 & 0011 \\ 0000 & 0001 \end{pmatrix}$$

$$\sigma_{K_1} \begin{pmatrix} 1101 & 0011 \\ 0000 & 0001 \end{pmatrix} = \begin{pmatrix} 1101 & 0011 \\ 0000 & 0001 \end{pmatrix} + \begin{pmatrix} 1000 & 0011 \\ 0101 & 0011 \end{pmatrix} = \begin{pmatrix} 0101 & 0000 \\ 0101 & 0010 \end{pmatrix}$$

$$\gamma \begin{pmatrix} 0101 & 0000 \\ 0101 & 0010 \end{pmatrix} = \begin{pmatrix} 0010 & 0011 \\ 0010 & 1111 \end{pmatrix}$$

$$\pi \begin{pmatrix} 0010 & 0011 \\ 0010 & 1111 \end{pmatrix} = \begin{pmatrix} 0010 & 0011 \\ 1111 & 0010 \end{pmatrix}$$

$$\sigma_{K_2} \begin{pmatrix} 0010 & 0011 \\ 1111 & 0010 \end{pmatrix} = \begin{pmatrix} 0010 & 0011 \\ 1111 & 0010 \end{pmatrix} + \begin{pmatrix} 1101 & 1011 \\ 1000 & 1000 \end{pmatrix} = \begin{pmatrix} 1111 & 1000 \\ 0111 & 1010 \end{pmatrix}$$

Luego $c_{[2]} = 1111 0111 1000 1010$ y, por tanto:

$$E_{dni}(0x1234567) = c = c_{[0]}c_{[1]}c_{[2]} = 0000 0000 0000 0001 0001 0110 0011 1011 1111 0111 1000 1010$$




\subsection{Apartado 2}
\textbf{Calcula $\text{E}_{dni}$(0x01234567) usando el modo CFB, r=11, y
vector de inicialización IV = 0x0001.}

Tenemos que $x_{[1]} = 0x0001=0000 0000 0000 0001$, que tiene 16 dígitos, y por tanto N=16.

Por otro lado, como $m=0x01234567$ tiene 32 dígitos, y 32 no es divisible por r=11, añado un 1 al final del mensaje, de forma que su longitud sea 33, divisible por 11. Ahora, dividimos el mensaje en secciones de 11 dígitos:

$$m = 0x01234567 = 0000 0001 0010 0011 0100 0101 0110 0111 1$$

Y por tanto $m_1=0000 0001 001$, $m_2=0 0011 0100 01$ y $m_3=01 0110 0111 1$, por tanto $l=3$.

La clave es la misma de ante, $k=0xAD60$, luego no se recalculan los $w_i$, con $i = 1, \cdots 11$.

Calculamos $E_K(x_{[1]})$:

$$\sigma_{K_0} \begin{pmatrix} 0000 & 0000 \\ 0000 & 0001 \end{pmatrix} = \begin{pmatrix} 0000 & 0000 \\ 0000 & 0001 \end{pmatrix} + \begin{pmatrix} 1010 & 0110 \\ 1101 & 0000 \end{pmatrix} = \begin{pmatrix} 1010 & 0110 \\ 1101 & 0001 \end{pmatrix}$$

$$\gamma \begin{pmatrix} 1010 & 0110 \\ 1101 & 0001 \end{pmatrix} = \begin{pmatrix} 1010 & 1011 \\ 1101 & 1000 \end{pmatrix}$$

$$\pi \begin{pmatrix} 1010 & 1011 \\ 1101 & 1000 \end{pmatrix} = \begin{pmatrix} 1010 & 10011 \\ 1000 & 1101 \end{pmatrix}$$

$$\theta \begin{pmatrix} 1010 & 1011 \\ 1000 & 1101 \end{pmatrix} = \begin{pmatrix} 0011 & 0010 \\ 0010 & 0011 \end{pmatrix}\begin{pmatrix} 1010 & 1011 \\ 1000 & 1101 \end{pmatrix} = \begin{pmatrix} 1110 & 0111 \\ 1100 & 0001 \end{pmatrix}$$

$$\sigma_{K_1} \begin{pmatrix} 1110 & 0111 \\ 1100 & 0001 \end{pmatrix} = \begin{pmatrix} 1110 & 0111 \\ 1100 & 0001 \end{pmatrix} + \begin{pmatrix} 1000 & 0011 \\ 0101 & 0011 \end{pmatrix} = \begin{pmatrix} 0110 & 0100 \\ 1001 & 0010 \end{pmatrix}$$

$$\gamma \begin{pmatrix} 0110 & 0100 \\ 1001 & 0010 \end{pmatrix} = \begin{pmatrix} 1011 & 0001 \\ 1110 & 1111 \end{pmatrix}$$

$$\pi \begin{pmatrix} 1011 & 0001 \\ 1110 & 1111 \end{pmatrix} = \begin{pmatrix} 1011 & 0001 \\ 1111 & 1110 \end{pmatrix}$$

$$\sigma_{K_2} \begin{pmatrix} 1011 & 0001 \\ 1111 & 1110 \end{pmatrix} = \begin{pmatrix} 1011 & 0001 \\ 1111 & 1110 \end{pmatrix} + \begin{pmatrix} 1101 & 1011 \\ 1000 & 1000 \end{pmatrix} = \begin{pmatrix} 0110 & 1010 \\ 0111 & 0110 \end{pmatrix}$$


luego $E_K(x_{[1]}) = 0110 0111 1010 0110$, y tenemos que:

$$\text{msb}_r (E_K(x_{[1]})) = 0110 01111 101$$
$$\text{lsb}_{N-r} (x_{[1]}) = 00001$$
$$c_{[1]} = m_1 \oplus \text{msb}_r(E_K(x_{[1]})) = 0000 0001 001 \oplus 0110 0111 101 = 0110 0110 100$$
$$x_{[2]} = \text{lsb}_{N-r} (x_{[1]}) || c_{[1]} = 0000 1011 0011 0100$$

Ahora $E_K(x_{[2]})$:

$$\sigma_{K_0} \begin{pmatrix} 0000 & 0011 \\ 1011 & 0100 \end{pmatrix} = \begin{pmatrix} 0000 & 0011 \\ 1011 & 0100 \end{pmatrix} + \begin{pmatrix} 1010 & 0110 \\ 1101 & 0000 \end{pmatrix} = \begin{pmatrix} 1010 & 0101 \\ 0110 & 0100 \end{pmatrix}$$

$$\gamma \begin{pmatrix} 1010 & 0101 \\ 0110 & 0100 \end{pmatrix} = \begin{pmatrix} 1010 & 0010 \\ 1011 & 0001 \end{pmatrix}$$

$$\pi \begin{pmatrix} 1010 & 0010 \\ 1011 & 0001 \end{pmatrix} = \begin{pmatrix} 1010 & 0010 \\ 0001 & 1011 \end{pmatrix}$$

$$\theta \begin{pmatrix} 1010 & 0010 \\ 0001 & 1011 \end{pmatrix} = \begin{pmatrix} 0011 & 0010 \\ 0010 & 0011 \end{pmatrix}\begin{pmatrix} 1010 & 0010 \\ 0001 & 1011 \end{pmatrix} = \begin{pmatrix} 1111 & 0011 \\ 0100 & 1010 \end{pmatrix}$$

$$\sigma_{K_1} \begin{pmatrix} 1111 & 0011 \\ 0100 & 1010 \end{pmatrix} = \begin{pmatrix} 1111 & 0011 \\ 0100 & 1010 \end{pmatrix} + \begin{pmatrix} 1000 & 0011 \\ 0101 & 0011 \end{pmatrix} = \begin{pmatrix} 0111 & 0000 \\ 0001 & 1001 \end{pmatrix}$$

$$\gamma \begin{pmatrix} 0111 & 0000 \\ 0001 & 1001 \end{pmatrix} = \begin{pmatrix} 0000 & 0011 \\ 1000 & 1110 \end{pmatrix}$$

$$\pi \begin{pmatrix} 0000 & 0011 \\ 1000 & 1110 \end{pmatrix} = \begin{pmatrix} 0000 & 0011 \\ 1110 & 1000 \end{pmatrix}$$

$$\sigma_{K_2} \begin{pmatrix} 0000 & 0011 \\ 1110 & 1000 \end{pmatrix} = \begin{pmatrix} 0000 & 0011 \\ 1110 & 1000 \end{pmatrix} + \begin{pmatrix} 1101 & 1011 \\ 1000 & 1000 \end{pmatrix} = \begin{pmatrix} 1101 & 1000 \\ 0110 & 0000 \end{pmatrix}$$

luego $E_K(x_{[2]}) = 1101 0110 1000 0000$, y tenemos que:

$$\text{msb}_r (E_K(x_{[2]})) = 1101 0110 100$$
$$\text{lsb}_{N-r} (x_{[2]}) = 10100$$
$$c_{[2]} = m_2 \oplus \text{msb}_r(E_K(x_{[2]})) = 0 0011 0100 01 \oplus 1101 0110 100 = 1100 1100 101$$
$$x_{[3]} = \text{lsb}_{N-r} (x_{[2]}) || c_{[2]} = 1010 0110 0110 0101$$

Ahora calculamos $E_K(x_{[3]})$:

$$\sigma_{K_0} \begin{pmatrix} 1010 & 0110 \\ 0110 & 0101 \end{pmatrix} = \begin{pmatrix} 1010 & 0110 \\ 0110 & 0101 \end{pmatrix} + \begin{pmatrix} 1010 & 0110 \\ 1101 & 0000 \end{pmatrix} = \begin{pmatrix} 0000 & 0000 \\ 1011 & 0101 \end{pmatrix}$$

$$\gamma \begin{pmatrix} 0000 & 0000 \\ 1011 & 0101 \end{pmatrix} = \begin{pmatrix} 0011 & 0011 \\ 0110 & 0010 \end{pmatrix}$$

$$\pi \begin{pmatrix} 0011 & 0011 \\ 0110 & 0010 \end{pmatrix} = \begin{pmatrix} 0011 & 0011 \\ 0010 & 0110 \end{pmatrix}$$

$$\theta \begin{pmatrix} 0011 & 0011 \\ 0010 & 0110 \end{pmatrix} = \begin{pmatrix} 0011 & 0010 \\ 0010 & 0011 \end{pmatrix}\begin{pmatrix} 0011 & 0011 \\ 0010 & 0110 \end{pmatrix} = \begin{pmatrix} 0001 & 1001 \\ 0000 & 1100 \end{pmatrix}$$

$$\sigma_{K_1} \begin{pmatrix} 0001 & 0011 \\ 0000 & 1100 \end{pmatrix} = \begin{pmatrix} 0001 & 0011 \\ 0000 & 1100 \end{pmatrix} + \begin{pmatrix} 1000 & 0011 \\ 0101 & 0011 \end{pmatrix} = \begin{pmatrix} 1001 & 1010 \\ 0101 & 1111 \end{pmatrix}$$

$$\gamma \begin{pmatrix} 1001 & 1010 \\ 0101 & 1111 \end{pmatrix} = \begin{pmatrix} 1110 & 1010 \\ 0010 & 0100 \end{pmatrix}$$

$$\pi \begin{pmatrix} 1110 & 1010 \\ 0010 & 0100 \end{pmatrix} = \begin{pmatrix} 1110 & 1010 \\ 0100 & 0010 \end{pmatrix}$$

$$\sigma_{K_2} \begin{pmatrix} 1110 & 1010 \\ 0100 & 0010 \end{pmatrix} = \begin{pmatrix} 1110 & 1010 \\ 0100 & 0010 \end{pmatrix} + \begin{pmatrix} 1101 & 1011 \\ 1000 & 1000 \end{pmatrix} = \begin{pmatrix} 0011 & 0001 \\ 1100 & 1010 \end{pmatrix}$$


luego $E_K(x_{[3]}) = 0011 1100 0001 1010$, y tenemos que:

$$\text{msb}_r (E_K(x_{[3]})) = 0011 1100 000$$
$$c_{[3]} = m_3 \oplus \text{msb}_r(E_K(x_{[3]})) = 0101 1001 111 \oplus 0011 1100 000 = 0110 0101 111$$

Por tanto $E_{dni} (0x01234567) = c = c_{[1]}c_{[2]}c_{[3]} = 0110 0110 100 1100 1100 101 0110 0101 111$.


\end{document}