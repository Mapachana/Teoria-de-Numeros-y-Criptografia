\documentclass[a4paper]{article}
%\usepackage[spanish]{babel}
\usepackage[utf8]{inputenc}
\usepackage{amsfonts}
\usepackage{amsmath}
\usepackage{graphicx}
\usepackage{float}
%\graphicspath{ {images/} }
\usepackage{hyperref}
\usepackage{enumerate}
\title {\fbox{\Huge{\textbf{Ejercicio Tema 3}}}}
\author {\fbox{Ana Buendía Ruiz-Azuaga}}
%\date {}
\begin{document}
\maketitle
%\tableofcontents


\section{Ejercicio Tema 3}

\textbf{Este ejercicio es individualizado. Cada uno parte del númerode su DNI, supongamos por ejemplo 12340987. Dividimos dicho número en dos bloques,1234 y 0987. Si alguno de ellos es menor que 1000,rotamos las cifras a la izquierda hasta obtener un número mayor o igual que 1000, en nuestro caso 9870 (si alguien tuviese como un bloque el 0000, que coja un número mayor que 1000 cualquiera a su elección.Sean p y q los primeros primos mayores o iguales que los bloques anteriores.  Concretamente,  en  el  ejemplo p=1237 y q=9871.  Sea n=pq y e el menor primo mayor o igual que 11 que es primo relativo con $\varphi(n)$. Sea $d=e^{-1}\mod \varphi(n)$}

Tomamos mi dni, que es 77770080, y por lo tanto lo dividimos en dos bloques, 7777 y 0080. Como 0080 es menor que 1000, rotamos las cifras dos veces, obteniendo 8000.

Por tanto, consideramos $p$ y $q$ los siguientes primos a estos números, resultando $p=7789$ y $q=8009$.

Definimos ahora $n=pq=62382101$, teniendo así que $\varphi(n)=62366304$, y como $e$ tomamos el menor primo mayor o igual que 11 que es primo relativo con $\varphi(n)$. Luego tenemos que $e=17$.

Finalmente, calculamos el inverso de $e$ módulo $\varphi(n)$, resultando así $d=29348849$.

\subsection{Apartado 1}

\textbf{Cifra el mensaje m=0xCAFE.}

Para cifrar el mensaje, siendo $m=0xCAFE$ calculamos (en mi caso en sagemath) $c=m^e \mod n$, obteniendo por tanto $c=0x3494740$.

\subsection{Apartado 2}
\textbf{Descifra el criptograma anterior.}

Para descifrar el mensaje, calculamos $m = c^d \mod n$, de donde resulta $m=0xCAFE$, que era el mensaje original.

\subsection{Apartado 3}
\textbf{Intenta factorizar n mediante el método P-1 de Pollard. Para ello llega, como máximo a b=8.}

Para factorizar, vamos a ir probando para los distintos valores de $b=1,\cdots ,8$. Comenzamos probando con $m=2$. Para hacer esto se ha usado el siguiente código:

\begin{verbatim}
m = 2
for b in range(1,9):
    print(b)
    pot = power_mod(m,factorial(b),n)
    print(n.gcd(pot-1))
\end{verbatim}

Podemos ver que el máximo común divisor siempre es 1, por lo que repetimos el proceso para el siguiente valor de $m$, que es $m=3$. Para automatizar este proceso, se ha implementado otro código también sagemath:

\begin{verbatim}
for m in range(1,1000):
    for b in range(1,9):
        pot = power_mod(m,factorial(b),n)
        mcd = n.gcd(pot-1)
        if(mcd != 1):
            print("MCD DISTINTO DE 1")
            print(m)
            print(b)
            print(mcd)
            break
\end{verbatim}

Al ejecutar este código, nos fijamos en la salida. La primera solución es el propio $n$, por lo que no nos sirve, pero el siguiente resultado sí. La salida resultante es:

\begin{verbatim}
MCD DISTINTO DE 1
1
1
62382101
MCD DISTINTO DE 1
233
3
7789
\end{verbatim}

de donde obtenemos que con $m=233$ y $b=3$, su máximo común divisor es 7789, luego uno de los factores de $n$ es 7789. Y, calculando $n/7789=8009$ obtenemos el otro factor.


\subsection{Apartado 4}
\textbf{Intenta factorizar n a partir de $\varphi(n)$.}

Sabemos que $\varphi(n)=62366304$, luego consideramos $(x-p)(x-q)=x^2-(p+q)x+n$, como $\varphi(n)= n+1-(p+q)$ tenemos $(x-p)(x-q)=x^2+(\varphi(n)-n-1)x+n$, luego p y q son las raices de este polinomio.

Calculamos por tanto las raíces del polinomio como se muestra:

\begin{verbatim}
x = var('x')
solve(x^2-(n+1-phi)*x+n, x)
\end{verbatim}

Por tanto, se obtienen como raíces del polinomio 7789 y 8009, que serían $p$ y $q$.

\subsection{Apartado 5}
\textbf{Intenta factorizar n a partir de e y d.}

Consideramos $k=ed-1=498930432$ y $a \in \{2, 3, 5, 7, 11, 13, 17, 19 \}$.

Como $a^k = a^{498930432} \equiv 1 \mod n$ para $a=2,3,5,7,11,13,17,19$ tomamos $k=k/2 = 249465216$.

Como $a^k = a^{249465216} \equiv 1 \mod n$ para $a=2,3,5,7,11,13,17,19$ tomamos $k=k/2 = 124732608$.

Como $a^k = a^{124732608} \equiv 1 \mod n$ para $a=2,3,5,7,11,13,17,19$ tomamos $k=k/2 = 62366304$.

Como $a^k = a^{62366304} \equiv 1 \mod n$ para $a=2,3,5,7,11,13,17,19$ tomamos $k=k/2 = 31183152$.

Como $a^k = a^{31183152} \equiv 1 \mod n$ para $a=2,3,5,7,11,13,17,19$ tomamos $k=k/2 = 15591576$.

Como $a^k = a^{15591576} \equiv 1 \mod n$ para $a=2,3,5,7,11,13,17,19$ tomamos $k=k/2 = 7795788$.

Como $3^{7795788} \mod n \equiv 11909382$ calculamos $(n,3^{7795788}-1)=7789$ y $\frac{n}{7789}=8009$ son los factores que buscamos.



\end{document}