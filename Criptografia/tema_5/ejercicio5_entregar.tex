\documentclass[a4paper]{article}
%\usepackage[spanish]{babel}
\usepackage[utf8]{inputenc}
\usepackage{amsfonts}
\usepackage{amsmath}
\usepackage{graphicx}
\usepackage{float}
%\graphicspath{ {images/} }
\usepackage{hyperref}
\usepackage{enumerate}
\title {\fbox{\Huge{\textbf{Ejercicio Tema 5}}}}
\author {\fbox{Ana Buendía Ruiz-Azuaga}}
%\date {}
\begin{document}
\maketitle
%\tableofcontents


\section{Ejercicio Tema 5}

\textbf{Sea $\mathbb{F}_{32}=\mathbb{F}_{2}[\xi]_{\xi^5+\xi^2+1}$. Cada uno de vosotros, de acuerdo a su número de DNI o similar, dispone de una curva elíptica sobre $\mathbb{F}_{32}$ y un punto base dados en el Cuadro 6.1.}

Mi dni es 77770080, luego tenemos que $77770080 \mod 32 \equiv 0$, por lo que de acuerdo al cuadro 6.1 la curva elíptica con la que vamos a trabajar es $E(\xi^3, \xi)$ y el punto $Q=(\xi^4+\xi+1, \xi^4+\xi^2+\xi)$.

\section{Apartado 1}

\textbf{Calcula, mediante el algoritmo de Shank o mediante el Algoritmo 9, $\log_{Q}O$}

Vamos a usar el algoritmo de Shank.

Acotamos $|E| \leq q+1+\lfloor 2\sqrt{q}\rfloor = 32+1+\lfloor 2\sqrt{32}\rfloor = 44$.

Luego $f=\lceil \sqrt{44} \rceil = 7$.

Construimos la tabla usando sagemath:

\begin{table}[!h]
\begin{center}
\begin{tabular}{ |c|c| } 
 \hline
 $Q$ & $(\xi^4+\xi+1, \xi^4+\xi^2+\xi)$  \\
 \hline
 $2Q$ & $(1,\xi^4+\xi^3+\xi^2+1)$  \\ 
 \hline
 $3Q$ & $(\xi^4+\xi^2+1, \xi^4+\xi^3)$  \\ 
 \hline
 $4Q$ & $(\xi+1,\xi+1)$  \\ 
 \hline
 $5Q$ & $(\xi^4+\xi^3+\xi^2+\xi, \xi^4+\xi^3+\xi^2+1)$  \\ 
 \hline
 $6Q$ & $(\xi^4+\xi^3+\xi, \xi^3+\xi)$  \\ 
 \hline
\end{tabular}
\end{center}
\end{table}



\section{Apartado 2}

\textbf{Para  tu  curva  y  tu  punto  base,  genera  un  par  de  claves  pública/privada para un protocolo ECDH.}

Res

\section{Apartado 3}

\textbf{Cifra el mensaje $(\xi^3+\xi^2+1,\xi^4+\xi^2)\in \mathbb{F}^2_{32}$ mediante el criptosistema de Menezes-Vanstone}

Res

\section{Apartado 4}

\textbf{Descifra el mensaje anterior.}

Res

\end{document}