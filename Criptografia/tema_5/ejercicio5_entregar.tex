\documentclass[a4paper]{article}
%\usepackage[spanish]{babel}
\usepackage[utf8]{inputenc}
\usepackage{amsfonts}
\usepackage{amsmath}
\usepackage{graphicx}
\usepackage{float}
%\graphicspath{ {images/} }
\usepackage{hyperref}
\usepackage{enumerate}
\title {\fbox{\Huge{\textbf{Ejercicio Tema 5}}}}
\author {\fbox{Ana Buendía Ruiz-Azuaga}}
%\date {}
\begin{document}
\maketitle
%\tableofcontents


\section{Ejercicio Tema 5}

\textbf{Sea $\mathbb{F}_{32}=\mathbb{F}_{2}[\xi]_{\xi^5+\xi^2+1}$. Cada uno de vosotros, de acuerdo a su número de DNI o similar, dispone de una curva elíptica sobre $\mathbb{F}_{32}$ y un punto base dados en el Cuadro 6.1.}

Mi dni es 77770080, luego tenemos que $77770080 \mod 32 \equiv 0$, por lo que de acuerdo al cuadro 6.1 la curva elíptica con la que vamos a trabajar es $E(\xi^3, \xi)$ y el punto $Q=(\xi^4+\xi+1, \xi^4+\xi^2+\xi)$.

\section{Apartado 1}

\textbf{Calcula, mediante el algoritmo de Shank o mediante el Algoritmo 9, $\log_{Q}O$}

Vamos a usar el algoritmo de Shank.

Acotamos $|E| \leq q+1+\lfloor 2\sqrt{q}\rfloor = 32+1+\lfloor 2\sqrt{32}\rfloor = 44$.

Luego $f=\lceil \sqrt{44} \rceil = 7$.

Construimos la tabla usando sagemath:

\begin{table}[!h]
\begin{center}
\begin{tabular}{ |c|c| } 
 \hline
 $0$ & $0$  \\
 \hline
 $Q$ & $(\xi^4+\xi+1, \xi^4+\xi^2+\xi)$  \\
 \hline
 $2Q$ & $(1,\xi^4+\xi^3+\xi^2+1)$  \\ 
 \hline
 $3Q$ & $(\xi^4+\xi^2+1, \xi^4+\xi^3)$  \\ 
 \hline
 $4Q$ & $(\xi+1,\xi+1)$  \\ 
 \hline
 $5Q$ & $(\xi^4+\xi^3+\xi^2+\xi, \xi^4+\xi^3+\xi^2+1)$  \\ 
 \hline
 $6Q$ & $(\xi^4+\xi^3+\xi, \xi^3+\xi)$  \\ 
 \hline
\end{tabular}
\end{center}
\end{table}

Ahora calculamos $-7Q=(\xi^3+\xi^2, \xi^2)$, que no está en la tabla, por lo que calculamos $2(-7Q)=(\xi^4+\xi^2+1, \xi^4+\xi^3)$, que se encuentra en la tabla, pues coincide con $3Q$, luego $2(-7Q)=3Q \Rightarrow O = 17Q \Rightarrow \log_Q O = 17$.




\section{Apartado 2}

\textbf{Para  tu  curva  y  tu  punto  base,  genera  un  par  de  claves  pública/privada para un protocolo ECDH.}

Partimos de nuestra curva elíptica y el punto base asignados.

Comenzamos calculando el orden de la curva elíptica, que es $|E|=hn$ con $h$ pequeño y $n$ primo. Tenemos que $|E|=34=2\cdot 17$ luego $h=2$ y $n=17$.

Ahora, Alice toma un número aleatorio $a$ con $2\leq a < n$ y calcula $P_a=aQ$ y envía a Bob $P_a$.

$$a= 13$$
$$P_a=(\xi+1, 0)$$

Ahora, Bob toma un número aleatorio $b$ con $2\leq b < n$ y calcula $P_b=bQ$ y envía a Alice $P_b$.
También calcula $bP_a$.

$$b= 10$$
$$P_b=(\xi^3+\xi^2, \xi^2)$$
$$bP_a=(\xi^4+\xi^3+\xi, \xi^4)$$

Finalmente Alice calcula $aP_b$:
$$aP_b=(\xi^4+\xi^3+\xi, \xi^4)$$

La clave compartida es $(ab)Q=aP_b=bP_a=(\xi^4+\xi^3+\xi, \xi^4)$.
Alice hace pública $(E,Q,P_a)$ y, análogamente, Bob hace pública $(E,Q,P_b)$.

\section{Apartado 3}

\textbf{Cifra el mensaje $(\xi^3+\xi^2+1,\xi^4+\xi^2)\in \mathbb{F}^2_{32}$ mediante el criptosistema de Menezes-Vanstone}

Vamos a usar la clave de Alice $a$. Comenzamos seleccionando aleatoriamente $k$ con $2\leq k < n$.

Calculamos $kQ$ y definimos $(x_0,y_0)= k(aQ)$. Si $x_0y_0=0$ tomamos otro $k$.

El cifrado es:

$$E(m_1,m_2)=(kQ, x_om_1, y_0m_2)$$.

En este caso $k=2$, luego tenemos

$$kQ=(1,\xi^4+\xi^3+\xi^2+1)$$
$$(x_0,y_0)=k(aQ)=2\cdot(13\cdot (\xi^4+\xi+1, \xi^4+\xi^2+\xi)=(\xi^3+\xi^2+\xi, \xi^4+\xi^3+1)$$

Esto es válido ya que $x_0\cdot y_0 = \xi^3+\xi$.

El mensaje a cifrar es $(m_1,m_2)=(\xi^3+\xi^2+1,\xi^4+\xi^2)$, luego tenemos que:

$$E(m_1,m_2)=((1, \xi^4+\xi^3+\xi^2+1), \xi^3+\xi^2, \xi)$$



\section{Apartado 4}

\textbf{Descifra el mensaje anterior.}

Para descifrar un criptograma $(C_1, c_2, c_3)$ Alice debe calcular $a(C_1)=a(kQ)=k(aQ)=(x_0,y_0)$ y

$$D(C_1,c_2,c_3)=(x_0^{-1}c_2, y_0^{-1}c_3)$$.

Tenemos que $a(C_1)=(x_0,y_0)=(\xi^3+\xi^2+\xi, \xi^4+\xi^3+1)$.

Luego

$$D(C_1,c_2,c_3)=(\xi^3+\xi^2+1, \xi^4+\xi^2)$$.



\end{document}