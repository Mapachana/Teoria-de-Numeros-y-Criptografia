\documentclass[a4paper]{article}
%\usepackage[spanish]{babel}
\usepackage[utf8]{inputenc}
\usepackage{amsfonts}
\usepackage{amsmath}
\usepackage{graphicx}
\usepackage{float}
%\graphicspath{ {images/} }
\usepackage{hyperref}
\usepackage{enumerate}
\title {\fbox{\Huge{\textbf{Ejercicio 8}}}}
\author {\fbox{Ana Buendía Ruiz-Azuaga}}
%\date {}
\begin{document}
\maketitle
%\tableofcontents


\section{Ejercicio 8}

\subsection{Apartado 1}

\textbf{Toma tu número n de la lista publicada para este ejercicio.}

$n = 2844871984646731064442373175299276800091$

\textbf{Pasa algunos tests de primalidad para ver si n es compuesto.}

Comenzamos pasando el test de Fermat para las bases $\{2,3,5,7,11\}$:

$$2^{n-1}\equiv 1\mod n$$
$$3^{n-1}\equiv 1\mod n$$
$$5^{n-1}\equiv 1\mod n$$
$$7^{n-1}\equiv 1\mod n$$
$$11^{n-1}\equiv 1\mod n$$

Luego $n$ es posible primo de Fermat para todas las bases.

Ahora vamos a pasarle el test de Miller-Rabin:

Como $\frac{n-1}{2}$ es impar, las a-sucesiones solo tendrán 2 términos:

\begin{verbatim}
La a-sucesion obtenida para la base 2 es:
[2844871984646731064442373175299276800090, 1]

La a-sucesion obtenida para la base 3 es:
[2844871984646731064442373175299276800090, 1]

La a-sucesion obtenida para la base 5 es:
[1, 1]

La a-sucesion obtenida para la base 7 es:
[1, 1]

La a-sucesion obtenida para la base 11 es:
[2844871984646731064442373175299276800090, 1]
\end{verbatim}

Como $2844871984646731064442373175299276800090\equiv -1 \mod n$ tenemos que $n$ pasa el test de Miller-Rabin para los 5 primeros primos, pues las sucesiones acaban en 1 y todo 1 va precedido de otro 1 o de -1.

\subsection{Apartado 2}
\textbf{En caso que tu n sea probable primo. Factoriza n + 1 encontrando certificadosde primalidad para los factores mayores de 10000.}

Aplicando $\rho$ de Polard a $n+1$ obtenemos:
$$n+1 = 2^2\cdot 31 \cdot 63929\cdot 600702031\cdot 352173733409\cdot 1696395339263.$$

Primero se han extraído los factores 2 y se ha aplicado el algoritmo a $711217996161682766110593293824819200023$, necesitando un total de más de 1000 iteraciones hasta descomponer el número.

Vamos a comprobar ahora mediante Lucas-Lehmer que los primos obtenidos mayores de $10000$ son, en efecto, primos.

\subsubsection{1696395339263}
Consideramos $p_1=1696395339263$, luego aplicando $\rho$ de Polard a $p_1-1$:

$$P_1-1=2 cdot 13 cdot 757 cdot 1621 \cdot 53171$$.

De nuevo, primero se ha extraído el factor 2 y se ha aplicado el método a $848197669631$, empleando 74 iteraciones en total.

Y por Lucas-Lehmer tenemos que $a=5$ es un elemento primitivo para $p_1$ porque $5^{p_1-1}\equiv 1\mod p_1$ y $5^{\frac{p_1-1}{p}}\not\equiv 1\mod p_1 $ para $p\in\{2, 13, 757, 1621, 53171\}$ pues:

\begin{verbatim}
5^(n-1)/ 2 = 1696395339262 mod n
5^(n-1)/ 13 = 1336486042586 mod n
5^(n-1)/ 757 = 1437998311805 mod n
5^(n-1)/ 1621 = 274109562190 mod n
5^(n-1)/ 53171 = 40822449061 mod n
\end{verbatim}

Consideramos ahora $p_2=53171$, luego aplicando $\rho$ de Polard a $p_2-1$:

$$p_2-1=2 cdot 5 cdot 13 cdot 409$$.

Se extra el factor 2 y se aplica el algoritmo a $26585$, necesitando 7 iteraciones en total.

Y por Lucas-Lehmer tenemos que $a=2$ es un elemento primitivo para $p_2$ porque $2^{p_2-1}\equiv 1\mod p_2$ y $2^{\frac{p_2-1}{p}}\not\equiv 1\mod p_2 $ para $p\in\{2, 5, 13, 409\}$ pues:

\begin{verbatim}
2^(n-1)/ 2 = 53170 mod n
2^(n-1)/ 5 = 25877 mod n
2^(n-1)/ 13 = 39138 mod n
2^(n-1)/ 409 = 30600 mod n
\end{verbatim}

\subsubsection{352173733409}
Consideramos $p_3=352173733409$, luego aplicando $\rho$ de Polard a $p_3-1$:

$$p_3-1=2^5 cdot 7 \cdot 349\cdot 4504883$$.

Primero se ha extraído el factor 2 y se aplica el método a $11005429169$, requiriendo 31 iteraciones en total.

Y por Lucas-Lehmer tenemos que $a=15$ es un elemento primitivo para $p_3$ porque $15^{p_3-1}\equiv 1\mod p_3$ y $15^{\frac{p_3-1}{p}}\not\equiv 1\mod p_3 $ para $p\in\{2, 7, 349, 4504883\}$ pues:

\begin{verbatim}
15^(n-1)/ 2 = 352173733408 mod n
15^(n-1)/ 7 = 72307373439 mod n
15^(n-1)/ 349 = 60311719490 mod n
15^(n-1)/ 4504883 = 173110241247 mod n
\end{verbatim}

Consideramos ahora $p_4=4504883$, luego aplicando $\rho$ de Polard a $p_4-1$:

$$p_4-1=2 cdot 2252441$$.

Primero se ha extraído el factor 2, y se va a comprobar que $2252441$ es primo.

Y por Lucas-Lehmer tenemos que $a=2$ es un elemento primitivo para $p_4$ porque $2^{p_4-1}\equiv 1\mod p_4$ y $2^{\frac{p_4-1}{p}}\not\equiv 1\mod p_4 $ para $p\in\{2, 2252441\}$ pues:

\begin{verbatim}
2^(n-1)/ 2 = 4504882 mod n
2^(n-1)/ 2252441 = 4 mod n
\end{verbatim}

Ahora comprobamos $p_5=2252441$, luego aplicando $\rho$ de Polard a $p_5-1$:

$$p_5-1=2^3 \cdot 5\cdot 56311$$.

De nuevo, se extraen los factor 2 y se aplica el método a $281555$, para lo que se necesitan 3 iteraciones en total.

Y por Lucas-Lehmer tenemos que $a=3$ es un elemento primitivo para $p_5$ porque $3^{p_5-1}\equiv 1\mod p_5$ y $3^{\frac{p_5-1}{p}}\not\equiv 1\mod p_5 $ para $p\in\{2, 5, 56311\}$ pues:

\begin{verbatim}
3^(n-1)/ 2 = 2252440 mod n
3^(n-1)/ 5 = 2075174 mod n
3^(n-1)/ 56311 = 1333115 mod n
\end{verbatim}

Finalmente consideramos $p_6=56311$, luego aplicando $\rho$ de Polard a $p_6-1$:

$$p_6-1=2 \cdot 3\cdot 5\cdot 1877$$.

Se extrae el factor 2 y se aplica el método a $28155$, necesitando este 4 iteraciones en total.

Y por Lucas-Lehmer tenemos que $a=6$ es un elemento primitivo para $p_6$ porque $6^{p_6-1}\equiv 1\mod p_6$ y $6^{\frac{p_6-1}{p}}\not\equiv 1\mod p_6 $ para $p\in\{2, 3,5,1877\}$ pues:

\begin{verbatim}
6^(n-1)/ 2 = 56310 mod n
6^(n-1)/ 3 = 14180 mod n
6^(n-1)/ 5 = 15485 mod n
6^(n-1)/ 1877 = 46171 mod n
\end{verbatim}

\subsubsection{600702031}

Consideramos $p_7=600702031$, luego aplicando $\rho$ de Polard a $p_7-1$:

$$p_7-1=2 \cdot 3^2\cdot 5\cdot 37\cdot 180391$$.

Primero sacamos el factor 2, y aplicamos el método a $300351015$, necesitando este número 9 iteraciones en total.

Y por Lucas-Lehmer tenemos que $a=3$ es un elemento primitivo para $p_7$ porque $3^{p_7-1}\equiv 1\mod p_7$ y $3^{\frac{p_7-1}{p}}\not\equiv 1\mod p_7 $ para $p\in\{2, 3,5,37, 180391\}$ pues:

\begin{verbatim}
3^(n-1)/ 2 = 600702030 mod n
3^(n-1)/ 3 = 267084186 mod n
3^(n-1)/ 5 = 455572699 mod n
3^(n-1)/ 37 = 27995379 mod n
3^(n-1)/ 180391 = 132564421 mod n
\end{verbatim}

Consideramos ahora $p_8=180391$, luego aplicando $\rho$ de Polard a $p_8-1$:

$$p_8-1=2 \cdot 3\cdot 5\cdot 7\cdot 859$$.

Sacamos el factor 2 y aplicamos el algoritmo a $90195$, que requiere de 6 iteraciones totales.

Y por Lucas-Lehmer tenemos que $a=7$ es un elemento primitivo para $p_8$ porque $7^{p_8-1}\equiv 1\mod p_8$ y $7^{\frac{p_8-1}{p}}\not\equiv 1\mod p_8 $ para $p\in\{2, 3,5,7, 859\}$ pues:

\begin{verbatim}
7^(n-1)/ 2 = 180390 mod n
7^(n-1)/ 3 = 83653 mod n
7^(n-1)/ 5 = 65181 mod n
7^(n-1)/ 7 = 129133 mod n
7^(n-1)/ 859 = 123807 mod n
\end{verbatim}

\subsubsection{63929}

Consideramos $p_9=63929$, luego aplicando $\rho$ de Polard a $p_9-1$:

$$p_9-1=2^3 \cdot 61\cdot 131$$.

Primero extraemos los factores 2 y aplicamos el algoritmo a $7991$, que necesita un total de 7 iteraciones.

Y por Lucas-Lehmer tenemos que $a=3$ es un elemento primitivo para $p_9$ porque $3^{p_9-1}\equiv 1\mod p_9$ y $3^{\frac{p_9-1}{p}}\not\equiv 1\mod p_9 $ para $p\in\{2, 61,131\}$ pues:

\begin{verbatim}
3^(n-1)/ 2 = 63928 mod n
3^(n-1)/ 61 = 46509 mod n
3^(n-1)/ 131 = 18863 mod n
\end{verbatim}


Luego hemos comprobado la correcta descomposición en primos de $n+1$:
$$n+1 = 2^2\cdot 31 \cdot 63929\cdot 600702031\cdot 352173733409\cdot 1696395339263.$$


\subsection{Apartado 3}

\textbf{Con P = 1, encuentra el menor Q natural mayor o igual que 2, tal que defina una s.L. que certifique la primalidad de n.}

Calculamos para cada $Q$ los valores de $U_{\frac{r}{p}}$ con $p$ siendo uno de los divisores de $r$ calculados en el apartado anterior.

\begin{verbatim}
Q: 2



U[2844871984646731064442373175299276800092] = 0, 
V[2844871984646731064442373175299276800092] = 4, 

U[2844871984646731064442373175299276800093] = 2

Factor: 2



U[1422435992323365532221186587649638400046] = 2401816336829289298745644029101509733749, 
V[1422435992323365532221186587649638400046] = 0, 

U[1422435992323365532221186587649638400047] = 2623344160738010181594008602200393266920

Factor: 31



U[91770064020862292401366876622557316132] = 1470916208500242344757444167006191571594, 
V[91770064020862292401366876622557316132] = 182521758238274725190305369534832625726, 

U[91770064020862292401366876622557316133] = 826718983369258534973874768270512098660

Factor: 63929



U[44500492493965666042678176966623548] = 1641957773477697682672844736651862712136, 
V[44500492493965666042678176966623548] = 2554432409192947819386117398691582182560, 

U[44500492493965666042678176966623549] = 2098195091335322751029481067671722447348

Factor: 600702031



U[4735912045961987207668310972132] = 2620483833716315257803347661958638569554, 
V[4735912045961987207668310972132] = 1027357537502469853234616045712241448265, 

U[4735912045961987207668310972133] = 401484693286027023297795266185801608864

Factor: 352173733409



U[8078035681732159367529888988] = 1364026177403439785504998557789553591298, 
V[8078035681732159367529888988] = 293516842564476323241204311857622760584, 

U[8078035681732159367529888989] = 828771509983958054373101434823588175941

Factor: 1696395339263



U[1677010021663162226328754084] = 1648554338650375930190024551175446944533, 
V[1677010021663162226328754084] = 1877548978459566186099302960294216193767, 

U[1677010021663162226328754085] = 1763051658554971058144663755734831569150
\end{verbatim}


Para P=1, Q=2 tenemos que ningún $U_{\frac{r}{p}}\not\equiv 0\mod n$ y $U_r\equiv 0\mod n$, luego tenemos que el rango de $n$ es $n+1$ y por tanto es primo.

\end{document}