\documentclass[a4paper]{article}
%\usepackage[spanish]{babel}
\usepackage[utf8]{inputenc}
\usepackage{amsfonts}
\usepackage{amsmath}
\usepackage{graphicx}
\usepackage{float}
%\graphicspath{ {images/} }
\usepackage{hyperref}
\usepackage{enumerate}
\title {\fbox{\Huge{\textbf{Ejercicio 8}}}}
\author {\fbox{Ana Buendía Ruiz-Azuaga}}
%\date {}
\begin{document}
\maketitle
%\tableofcontents


\section{Ejercicio 8}

\subsection{Apartado 1}

\textbf{Toma tu número n de la lista publicada para este ejercicio.}

$n = 2844871984646731064442373175299276800091$

\textbf{Pasa algunos tests de primalidad para ver si n es compuesto.}

\subsection{Apartado 2}
\textbf{En caso que tu n sea probable primo. Factoriza n + 1 encontrando certificadosde primalidad para los factores mayores de 10000.}


\subsection{Apartado 3}

\textbf{Con P = 1, encuentra el menor Q natural mayor o igual que 2, tal que defina una s.L. que certifique la primalidad de n.}

\end{document}