\documentclass[a4paper]{article}
%\usepackage[spanish]{babel}
\usepackage[utf8]{inputenc}
\usepackage{amsfonts}
\usepackage{amsmath}
\usepackage{graphicx}
\usepackage{float}
%\graphicspath{ {images/} }
\usepackage{hyperref}
\usepackage{enumerate}
\title {\fbox{\Huge{\textbf{Ejercicio 9}}}}
\author {\fbox{Ana Buendía Ruiz-Azuaga}}
%\date {}
\begin{document}
\maketitle
%\tableofcontents


\section{Ejercicio 9}

\subsection{Apartado 1}

\textbf{Toma n tu número publicado para el ejer_ 2. Escríbe n en base 2, usa esas cifras para definir un polinomio, f(x), donde tu bit más significativo defina el grado del polinomio n, el siguiente bit va multiplicado por $x^{n-1}$ y sucesivamente hasta que el bit menos significativo sea el término independiente.El polinomio que obtienes es universal en el sentido de que tiene coeficientes en cualquier anillo}

\textbf{Sea f(x) el polinomio que obtienes con coeficientes en $\mathbb{Z}$}

$n = $

\textbf{Toma g(x) = f(x) mod 2 y halla el menor cuerpo de característica 2 que contenga a todas las raíces de g. ¿ Qué deduces sobre la irreducibilidad de g(x) en $\mathbb{Z}_2[x]$ ?.}

\subsection{Apartado 2}
\textbf{Extrae la parte libre de cuadrados de g(x) y le calculas su matriz de Berlekamp por columnas. Resuelve el s.l. (B-Id)X=0.}



\subsection{Apartado 3}
\textbf{Aplica Berlekamp si es necesario recursivamente para hallar la descomposiciónen irreducibles de g(x) en $\mathbb{Z}_2[x]$.}



\subsection{Apartado 4}
\textbf{Haz lo mismo para hallar la descomposición en irreducibles de f(x) mod 3}



\subsection{Apartado 5}
\textbf{¿ Qué deduces sobre la reducibilidad de f(x) en $\mathbb{Z}[x]$ ?}




\end{document}