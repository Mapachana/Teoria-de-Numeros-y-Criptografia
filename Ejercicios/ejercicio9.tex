\documentclass[a4paper]{article}
%\usepackage[spanish]{babel}
\usepackage[utf8]{inputenc}
\usepackage{amsfonts}
\usepackage{amsmath}
\usepackage{graphicx}
\usepackage{float}
%\graphicspath{ {images/} }
\usepackage{hyperref}
\usepackage{enumerate}
\title {\fbox{\Huge{\textbf{Ejercicio 9}}}}
\author {\fbox{Ana Buendía Ruiz-Azuaga}}
%\date {}
\begin{document}
\maketitle
%\tableofcontents


\section{Ejercicio 9}

\subsection{Apartado 1}

\textbf{Toma n tu número publicado para el ejercicio 2. Escríbe n en base 2, usa esas cifras para definir un polinomio, f(x), donde tu bit más significativo defina el grado del polinomio n, el siguiente bit va multiplicado por $x^{n-1}$ y sucesivamente hasta que el bit menos significativo sea el término independiente.El polinomio que obtienes es universal en el sentido de que tiene coeficientes en cualquier anillo}

\textbf{Sea f(x) el polinomio que obtienes con coeficientes en $\mathbb{Z}$}

$n = 77770081$

Tenemos que $n$ en base 2 es 100101000101010110101100001, luego

$$f(x)=x^{26} + x^{23} + x^{21} + x^{17} + x^{15} + x^{13} + x^{11} + x^{10} + x^8 + x^6 + x^5 + 1.$$

\textbf{Toma g(x) = f(x) mod 2 y halla el menor cuerpo de característica 2 que contenga a todas las raíces de g. ¿ Qué deduces sobre la irreducibilidad de g(x) en $\mathbb{Z}_2[x]$ ?.}

Definimos $g(x)$ como $g(x)=f(x) \mod 2$, luego 
$$g(x)= x^{26} + x^{23} + x^{21} + x^{17} + x^{15} + x^{13} + x^{11} + x^{10} + x^8 + x^6 + x^5 + 1.$$

El menor cuerpo de característica 2 que contiene a todas las raíces de $g$ es $F_{2^{46}}=F_{2^{2\cdot 23}}$. Como tenemos que $46>26$ entonces $g(x)$ es reducible en $\mathbb{Z}_2[x]$. Además, sabemos que los factores en los que se descomponga $g$ serán de grado un divisor de $46$, esto es, $1$, $2$ o $23$. 


\subsection{Apartado 2}
\textbf{Extrae la parte libre de cuadrados de g(x) y le calculas su matriz de Berlekamp por columnas. Resuelve el s.l. (B-Id)X=0.}

Tenemos que el máximo común divisor de $g$ y su derivada es $(g, g')=1$, luego $g$ es libre de cuadrados. Pasamos por tanto a calcular su matriz de Berlekamp por columnas.

para ello, comenzamos por calcular $x^{2i} \mod g$ con $0\leq i < 26$

\begin{verbatim}
[0] x^2i mod f = 1
[1] x^2i mod f = x^2
[2] x^2i mod f = x^4
[3] x^2i mod f = x^6
[4] x^2i mod f = x^8
[5] x^2i mod f = x^10
[6] x^2i mod f = x^12
[7] x^2i mod f = x^14
[8] x^2i mod f = x^16
[9] x^2i mod f = x^18
[10] x^2i mod f = x^20
[11] x^2i mod f = x^22
[12] x^2i mod f = x^24
[13] x^2i mod f = x^23 + x^21 + x^17 + x^15 + x^13 + x^11 + x^10 + x^8 + x^6 +
 x^5 + 1
[14] x^2i mod f = x^25 + x^23 + x^19 + x^17 + x^15 + x^13 + x^12 + x^10 + x^8 +
 x^7 + x^2
[15] x^2i mod f = x^25 + x^24 + x^22 + x^21 + x^19 + x^18 + x^17 + x^16 + x^15 +
 x^11 + x^10 + x^7 + x^6 + x^4 + x
[16] x^2i mod f = x^22 + x^20 + x^19 + x^16 + x^15 + x^14 + x^10 + x^7 + x^6 +
 x^5 + x^3 + x + 1
[17] x^2i mod f = x^24 + x^22 + x^21 + x^18 + x^17 + x^16 + x^12 + x^9 + x^8 +
 x^7 + x^5 + x^3 + x^2
[18] x^2i mod f = x^24 + x^21 + x^20 + x^19 + x^18 + x^17 + x^15 + x^14 + x^13
 + x^9 + x^8 + x^7 + x^6 + x^4 + 1
[19] x^2i mod f = x^22 + x^20 + x^19 + x^16 + x^13 + x^9 + x^5 + x^2 + 1
[20] x^2i mod f = x^24 + x^22 + x^21 + x^18 + x^15 + x^11 + x^7 + x^4 + x^2
[21] x^2i mod f = x^24 + x^21 + x^20 + x^15 + x^11 + x^10 + x^9 + x^8 + x^5 +
 x^4 + 1
[22] x^2i mod f = x^22 + x^21 + x^15 + x^12 + x^8 + x^7 + x^5 + x^2 + 1
[23] x^2i mod f = x^24 + x^23 + x^17 + x^14 + x^10 + x^9 + x^7 + x^4 + x^2
[24] x^2i mod f = x^25 + x^23 + x^21 + x^19 + x^17 + x^16 + x^15 + x^13 + x^12 
+ x^10 + x^9 + x^8 + x^5 + x^4 + 1
[25] x^2i mod f = x^25 + x^24 + x^23 + x^22 + x^21 + x^19 + x^17 + x^16 + x^15
 + x^10 + x^9 + x^2 + x
\end{verbatim}

Y por tanto, la matriz de Berlekamp B, cuyas filas vienen dadas por los coeficientes de los polinomios calculados resulta:

\begin{verbatim}
[1 0 0 0 0 0 0 0 0 0 0 0 0 0 0 0 0 0 0 0 0 0 0 0 0 0]
[0 0 1 0 0 0 0 0 0 0 0 0 0 0 0 0 0 0 0 0 0 0 0 0 0 0]
[0 0 0 0 1 0 0 0 0 0 0 0 0 0 0 0 0 0 0 0 0 0 0 0 0 0]
[0 0 0 0 0 0 1 0 0 0 0 0 0 0 0 0 0 0 0 0 0 0 0 0 0 0]
[0 0 0 0 0 0 0 0 1 0 0 0 0 0 0 0 0 0 0 0 0 0 0 0 0 0]
[0 0 0 0 0 0 0 0 0 0 1 0 0 0 0 0 0 0 0 0 0 0 0 0 0 0]
[0 0 0 0 0 0 0 0 0 0 0 0 1 0 0 0 0 0 0 0 0 0 0 0 0 0]
[0 0 0 0 0 0 0 0 0 0 0 0 0 0 1 0 0 0 0 0 0 0 0 0 0 0]
[0 0 0 0 0 0 0 0 0 0 0 0 0 0 0 0 1 0 0 0 0 0 0 0 0 0]
[0 0 0 0 0 0 0 0 0 0 0 0 0 0 0 0 0 0 1 0 0 0 0 0 0 0]
[0 0 0 0 0 0 0 0 0 0 0 0 0 0 0 0 0 0 0 0 1 0 0 0 0 0]
[0 0 0 0 0 0 0 0 0 0 0 0 0 0 0 0 0 0 0 0 0 0 1 0 0 0]
[0 0 0 0 0 0 0 0 0 0 0 0 0 0 0 0 0 0 0 0 0 0 0 0 1 0]
[1 0 0 0 0 1 1 0 1 0 1 1 0 1 0 1 0 1 0 0 0 1 0 1 0 0]
[0 0 1 0 0 0 0 1 1 0 1 0 1 1 0 1 0 1 0 1 0 0 0 1 0 1]
[0 1 0 0 1 0 1 1 0 0 1 1 0 0 0 1 1 1 1 1 0 1 1 0 1 1]
[1 1 0 1 0 1 1 1 0 0 1 0 0 0 1 1 1 0 0 1 1 0 1 0 0 0]
[0 0 1 1 0 1 0 1 1 1 0 0 1 0 0 0 1 1 1 0 0 1 1 0 1 0]
[1 0 0 0 1 0 1 1 1 1 0 0 0 1 1 1 0 1 1 1 1 1 0 0 1 0]
[1 0 1 0 0 1 0 0 0 1 0 0 0 1 0 0 1 0 0 1 1 0 1 0 0 0]
[0 0 1 0 1 0 0 1 0 0 0 1 0 0 0 1 0 0 1 0 0 1 1 0 1 0]
[1 0 0 0 1 1 0 0 1 1 1 1 0 0 0 1 0 0 0 0 1 1 0 0 1 0]
[1 0 1 0 0 1 0 1 1 0 0 0 1 0 0 1 0 0 0 0 0 1 1 0 0 0]
[0 0 1 0 1 0 0 1 0 1 1 0 0 0 1 0 0 1 0 0 0 0 0 1 1 0]
[1 0 0 0 1 1 0 0 1 1 1 0 1 1 0 1 1 1 0 1 0 1 0 1 0 1]
[0 1 1 0 0 0 0 0 0 1 1 0 0 0 0 1 1 1 0 1 0 1 1 1 1 1]
\end{verbatim}

Ahora que tenemos construida B, vamos a resolver (B-Id)X=0.

El rango de B-Id es 23, luego la dimensión de $V_1=26-23=3$. Por tanto, hay $3$ soluciones, que son:

\begin{verbatim}
[1 0 0 0 0 0 0 0 0 0 0 0 0 0 0 0 0 0 0 0 0 0 0 0 0 0]
[0 1 0 1 1 1 1 0 0 0 1 0 1 0 1 0 1 0 1 1 0 0 0 1 1 0]
[0 0 1 0 0 1 0 1 0 0 0 0 1 1 0 0 1 1 0 0 1 0 0 0 0 1]
\end{verbatim}

Luego obtenemos los polinomios:
$$f_1(x) = 1$$
$$f_2(x) = x+x^3+x^4+x^5+x^6+x^{10}+x^{12}+x^{14}+x^{16}+x^{18}+x^{19}+x^{23}+x^{24}$$
$$f_3(x) = x^2+x^5+x^7+x^{12}+x^{13}+x^{16}+x^{17}+x^{20}+x^{25}$$

donde $f_2$ y $f_3$ son g-reductores.

\subsection{Apartado 3}
\textbf{Aplica Berlekamp si es necesario recursivamente para hallar la descomposiciónen irreducibles de g(x) en $\mathbb{Z}_2[x]$.}

Comenzamos trabajando con $f_2$, para el que calculamos $(g, f_2) =h_1$ y $(g, f_2+1)=h_2$ y comprobamos que son libres de cuadrados:

\begin{verbatim}
(g, f2) = x^2 + x + 1
(g, f2+1) = x^24 + x^23 + x^19 + x^18 + x^16 + x^14 + x^12 + x^10 + x^6 + x^5 +
 x^4 + x^3 + x + 1
\end{verbatim}

Luego tenemos que $g=h_1\cdot h_2$.

Comprobamos que tanto $h_1$ como $h_2$ son libres de cuadrados, pues $(h_1,h_1')=1$ y $(h_2,h_2')=1$.

El menor cuerpo  de característica que contiene las raíces de $h_1$ es $F_{2^{2}}$, y como $h_1$ tiene grado 2, por tanto es irreducible.

Como $h_2$ tiene grado 24 y el menor cuerpo de característica 2 que contiene sus raíces es $F_{2^{23}}$, con 23<24, no sabemos si debe ser irreducible. En efecto, tenemos que $h_2$ tiene 1 como raíz y por tanto descompone como 
$$h_2=(x+1)\cdot (x^{23}+x^{18}+x^{15}+x^{14}+x^{11}+x^{10}+x^5+x^3+1)$$

Luego la descomposición en irreducibles de $g$ es:
$$g=(x+1)\cdot (x^2+x+1) \cdot (x^{23}+x^{18}+x^{15}+x^{14}+x^{11}+x^{10}+x^5+x^3+1)$$

Hacemos lo análogo con $f_3$, considerando $(g, f_3) =h_3$ y $(g, f_3+1)=h_4$ 

\begin{verbatim}
(g, f3) = x^23 + x^18 + x^15 + x^14 + x^11 + x^10 + x^5 + x^3 + 1
(g, f3+1) = x^3 + 1
\end{verbatim}

Obteniendo por tanto que $g=h_3\cdot h_4$

Comprobamos que tanto $h_3$ como $h_4$ son libres de cuadrados, pues $(h_3,h_3')=1$ y $(h_4,h_4')=1$.

El menor cuerpo de característica 2 que contiene las raíces de $h_3$ es $F_{2^{23}}$, y como $h_3$ tiene grado 23 tenemos que es irreducible.

De nuevo, observamos que $h_4$ tiene grado 4 y el menor cuerpo de característica 2 que contiene sus raíces es $F_{2^{2}}$, luego no sabemos si es irreducible, si lo fuera, tendría un factor de grado 1 y otro de grado 2, y en efecto $h_4=(x+1)\cdot x^2+x+1)$, pues 1 es raíz suyo, y la descomposición de $g$ en irreducibles resulta:

$$g=(x+1)\cdot (x^2+x+1) \cdot (x^{23}+x^{18}+x^{15}+x^{14}+x^{11}+x^{10}+x^5+x^3+1)$$







\subsection{Apartado 4}
\textbf{Haz lo mismo para hallar la descomposición en irreducibles de f(x) mod 3}

Nota: En este apartado se va a usar y en lugar de x para los polinomios para que vaya de acuerdo al código implementado.

Tenemos que $h = f \mod 3 = y^{26} + y^{23} + y^{21} + y^{17} + y^{15} + y^{13} + y^{11} + y^{10} + y^8 + y^6 + y^5 + 1$

Tenemos que $(h, h')=1$, luego $h$ es libre de cuadrados.

El menor cuerpo de característica 3 que contiene las raíces de $h$ es $F_{3^{117}}=F_{3^{3^2\cdot 13}}$, y como $117>26$ tenemos que $h$ es reducible, con factores de grado 1, 3, 9 o 13.

Calculamos ahora la matriz B por columnas, para lo que calculamos de nuevo primero $y^{2i}\mod h$ con $0\leq i < 26$:

\begin{verbatim}
[0] x^2i mod f = 1
[1] x^2i mod f = y^3
[2] x^2i mod f = y^6
[3] x^2i mod f = y^9
[4] x^2i mod f = y^12
[5] x^2i mod f = y^15
[6] x^2i mod f = y^18
[7] x^2i mod f = y^21
[8] x^2i mod f = y^24
[9] x^2i mod f = 2*y^24 + 2*y^22 + 2*y^18 + 2*y^16 + 2*y^14 + 2*y^12 + 2*y^11 +
 2*y^9 + 2*y^7 + 2*y^6 + 2*y
[10] x^2i mod f = 2*y^25 + y^24 + y^22 + 2*y^21 + 2*y^19 + y^18 + 2*y^17 +
 y^16 + 2*y^15 + y^11 + 2*y^10 + y^7 + y^6 + 2*y^4 + y
[11] x^2i mod f = 2*y^25 + y^24 + y^23 + y^22 + y^21 + 2*y^20 + 2*y^19 + y^18 +
 y^17 + 2*y^16 + y^15 + 2*y^11 + 2*y^10 + y^8 + 2*y^7 + 2*y^6 + y^4 + y^2 + 2*y
[12] x^2i mod f = 2*y^25 + 2*y^23 + y^22 + y^20 + 2*y^16 + y^14 + 2*y^13 + 2*y^11 +
 2*y^10 + y^9 + y^7 + y^6 + 2*y^4 + y^2 + 2*y + 2
[13] x^2i mod f = 2*y^25 + y^21 + 2*y^16 + 2*y^15 + 2*y^14 + y^13 + 2*y^12 +
 y^11 + y^9 + 2*y^8 + y^6 + 2*y^5 + 2*y^4 + 2*y^3 + y^2 + 1
[14] x^2i mod f = y^25 + y^24 + y^23 + 2*y^18 + y^16 + y^14 + y^13 + 2*y^12 + 2*y^11 +
 y^10 + y^9 + 2*y^6 + y^5 + y^3 + y^2
[15] x^2i mod f = 2*y^25 + 2*y^24 + y^23 + 2*y^22 + y^21 + 2*y^18 + 2*y^17 + y^14 +
 2*y^13 + 2*y^12 + y^11 + y^10 + y^9 + 2*y^8 + y^7 + 2*y^6 + 2*y^2 + 2*y + 2
[16] x^2i mod f = 2*y^24 + y^22 + y^21 + 2*y^20 + y^19 + y^18 + y^17 + 2*y^15 + 2*y^14 +
 y^13 + 2*y^11 + y^10 + 2*y^7 + y^5 + 2*y^4 + 2*y^3 + y^2 + y + 2
[17] x^2i mod f = y^25 + 2*y^24 + 2*y^23 + 2*y^22 + y^21 + y^20 + 2*y^17 + 2*y^16 +
 y^13 + y^12 + y^11 + 2*y^10 + y^9 + y^8 + y^5 + y^4 + 2*y^3 + y
[18] x^2i mod f = y^25 + 2*y^24 + y^23 + y^22 + y^21 + 2*y^20 + y^19 + y^18 + 2*y^16 +
 y^15 + 2*y^14 + 2*y^13 + y^12 + y^9 + y^8 + y^7 + y^6 + y^5 + y^4 + 2*y^2 + y + 1
[19] x^2i mod f = 2*y^24 + 2*y^22 + y^19 + 2*y^18 + 2*y^15 + y^14 + y^13 + y^12 + y^11 +
 2*y^10 + 2*y^9 + 2*y^8 + y^7 + y^5 + y^4 + y^3 + 2*y^2 + y + 2
[20] x^2i mod f = 2*y^25 + y^24 + 2*y^22 + 2*y^21 + y^17 + 2*y^16 + y^15 + 2*y^14 +
 2*y^13 + y^10 + y^9 + y^8 + 2*y^7 + 2*y^6 + 2*y^5 + y^4 + 2*y^3 + y
[21] x^2i mod f = y^24 + y^23 + 2*y^22 + y^20 + y^16 + y^15 + 2*y^14 + 2*y^13 + y^12 +
 y^9 + y^7 + y^6 + y^4 + y^2 + 2*y
[22] x^2i mod f = 2*y^25 + 2*y^24 + 2*y^22 + 2*y^21 + y^19 + y^17 + y^16 + 2*y^14 + 
2*y^13 + y^11 + 2*y^8 + y^6 + 2*y^4 + 2*y + 2
[23] x^2i mod f = y^23 + 2*y^22 + y^20 + 2*y^19 + y^18 + y^15 + 2*y^14 + y^13 + 2*y^12 +
 y^10 + 2*y^9 + y^8 + y^7 + y^6 + 2*y^4 + 2*y^3 + y^2 + y
[24] x^2i mod f = 2*y^25 + 2*y^22 + y^18 + y^17 + y^16 + y^15 + 2*y^12 + y^9 + 2*y^8 +
 2*y^7 + y^6 + y^4 + 2
[25] x^2i mod f = y^23 + y^21 + y^20 + 2*y^19 + y^18 + y^17 + y^13 + 2*y^12 + 2*y^11 +
 y^9 + y^8 + 2*y^7 + 2*y^3 + y^2
\end{verbatim}

Y B, con sus filas formadas por los coeficientes de estos polinomios, resulta:

\begin{verbatim}
[1 0 0 0 0 0 0 0 0 0 0 0 0 0 0 0 0 0 0 0 0 0 0 0 0 0]
[0 0 0 1 0 0 0 0 0 0 0 0 0 0 0 0 0 0 0 0 0 0 0 0 0 0]
[0 0 0 0 0 0 1 0 0 0 0 0 0 0 0 0 0 0 0 0 0 0 0 0 0 0]
[0 0 0 0 0 0 0 0 0 1 0 0 0 0 0 0 0 0 0 0 0 0 0 0 0 0]
[0 0 0 0 0 0 0 0 0 0 0 0 1 0 0 0 0 0 0 0 0 0 0 0 0 0]
[0 0 0 0 0 0 0 0 0 0 0 0 0 0 0 1 0 0 0 0 0 0 0 0 0 0]
[0 0 0 0 0 0 0 0 0 0 0 0 0 0 0 0 0 0 1 0 0 0 0 0 0 0]
[0 0 0 0 0 0 0 0 0 0 0 0 0 0 0 0 0 0 0 0 0 1 0 0 0 0]
[0 0 0 0 0 0 0 0 0 0 0 0 0 0 0 0 0 0 0 0 0 0 0 0 1 0]
[0 2 0 0 0 0 2 2 0 2 0 2 2 0 2 0 2 0 2 0 0 0 2 0 2 0]
[0 1 0 0 2 0 1 1 0 0 2 1 0 0 0 2 1 2 1 2 0 2 1 0 1 2]
[0 2 1 0 1 0 2 2 1 0 2 2 0 0 0 1 2 1 1 2 2 1 1 1 1 2]
[2 2 1 0 2 0 1 1 0 1 2 2 0 2 1 0 2 0 0 0 1 0 1 2 0 2]
[1 0 1 2 2 2 1 0 2 1 0 1 2 1 2 2 2 0 0 0 0 1 0 0 0 2]
[0 0 1 1 0 1 2 0 0 1 1 2 2 1 1 0 1 0 2 0 0 0 0 1 1 1]
[2 2 2 0 0 0 2 1 2 1 1 1 2 2 1 0 0 2 2 0 0 1 2 1 2 2]
[2 1 1 2 2 1 0 2 0 0 1 2 0 1 2 2 0 1 1 1 2 1 1 0 2 0]
[0 1 0 2 1 1 0 0 1 1 2 1 1 1 0 0 2 2 0 0 1 1 2 2 2 1]
[1 1 2 0 1 1 1 1 1 1 0 0 1 2 2 1 2 0 1 1 2 1 1 1 2 1]
[2 1 2 1 1 1 0 1 2 2 2 1 1 1 1 2 0 0 2 1 0 0 2 0 2 0]
[0 1 0 2 1 2 2 2 1 1 1 0 0 2 2 1 2 1 0 0 0 2 2 0 1 2]
[0 2 1 0 1 0 1 1 0 1 0 0 1 2 2 1 1 0 0 0 1 0 2 1 1 0]
[2 2 0 0 2 0 1 0 2 0 0 1 0 2 2 0 1 1 0 1 0 2 2 0 2 2]
[0 1 1 2 2 0 1 1 1 2 1 0 2 1 2 1 0 0 1 2 1 0 2 1 0 0]
[2 0 0 0 1 0 1 2 2 1 0 0 2 0 0 1 1 1 1 0 0 0 2 0 0 2]
[0 0 1 2 0 0 0 2 1 1 0 2 2 1 0 0 0 1 1 2 1 1 0 1 0 0]
\end{verbatim}

Como (B-Id) tiene rango 22, entonces $26-22=4$ tendremos 4 soluciones.
Y resolvemos el sistema (B-Id)X=0, obteniendo como soluciones:

\begin{verbatim}
[1 0 0 0 0 0 0 0 0 0 0 0 0 0 0 0 0 0 0 0 0 0 0 0 0 0]
[0 2 0 1 1 1 1 2 1 1 0 1 1 0 1 0 2 2 0 0 1 1 2 0 1 0]
[0 0 1 2 0 2 1 0 2 0 1 1 2 1 0 1 0 1 2 1 1 2 2 2 0 0]
[0 0 0 0 1 2 0 1 0 1 2 0 1 0 1 2 2 2 2 1 0 0 2 0 0 1]
\end{verbatim}

Que son los siguientes polinomios:

$$f_1 = 1$$
$$f_2 = 2*y+y^3+y^4+y^5+y^6+2*y^7+y^8+y^9+y^{11}+y^{12}+y^{14}+2*y^{16}+2*y^{17}+y^{20}+y^{21}+2*y^{22}+y^{24}$$
$$f_3 = y^2+2*y^3+2*y^5+y^6+2*y^8+y^{10}+y^{11}+2*y^{12}+y^{13}+y^{15}+y^{17}+2*y^{18}+y^{19}+y^{20}+2*y^{21}+2*y^{22}+2*y^{23}$$
$$f_4 = y^4+2*y^5+y^7+y^9+2*y^{10}+y^{12}+y^{14}+2*y^{15}+2*y^{16}+2*y^{17}+2*y^{18}+y^{19}+2*y^{22}+y^{25}$$

Calculamos los mcd correspondientes a $f_2$, siendo $h_1=(g,f_2)$, $h_2=(g,f_2+1)$ y $h_3=(g,f_2+2)$:

\begin{verbatim}
(g, f2) = y^3 + 2*y^2 + 1
(g, f2+1) = y^13 + 2*y^12 + y^11 + y^10 + 2*y^9 + 2*y^8 + 2*y^7 + y^5 + y^4 + y^2 + 2
(g, f2+2) = y^10 + 2*y^9 + 2*y^8 + y^5 + 2*y^4 + y^3 + y^2 + 2
\end{verbatim}

Obtenemos por tanto que $h=h_1\cdot h_2\cdot h_3$, siendo estos tres polinomios libres de cuadrados, y como el mínimo cuerpo de característica 3 que contiene las raíces de $h_1$ es $F_{3^{3}}$, coincidiendo 3 con su grado, se tiene que es irreducible. Análogamente se tiene que para $h_2$ es $F_{3^{13}}$ y su grado es 13, y por tanto es irreducible. Mientras para $h_3$ es $F_{3^{9}}$, siendo 9 distinto de su grado, por lo que es reducible y es fácil comprobar que tiene 1 como raíz simple, luego se descompone en $h_3=(x+2)\cdot (y^9+2y^7+2y^6+2y^5+2y^3+y+1)$, luego la descomposición de $h$ en irreducibles es:

$h = (y^3 + 2y^2 + 1)\cdot (y^{13} + 2y^{12} + y^{11} + y^{10} + 2*y^9 + 2*y^8 + 2*y^7 + y^5 + y^4 + y^2 + 2)\cdot (y+2)\cdot (y^9 + 2y^7 + 2y^6 + 2y^5 + 2y^3 + y + 1)$

Para $f_3$, siendo $h_4=(g,f_3)$, $h_5=(g,f_3+1)$ y $h_6=(g,f_3+2)$:

\begin{verbatim}
(g, f3) = y^16 + y^15 + 2*y^14 + y^13 + y^11 + y^10 + 2*y^7 + 2*y^6 +
 2*y^5 + 2*y^3 + 2*y^2 + 2
(g, f3+1) = y^9 + 2*y^7 + 2*y^6 + 2*y^5 + 2*y^3 + y + 1
(g, f3+2) = y + 2
\end{verbatim}

Obtenemos por tanto que $h=h_4\cdot h_5\cdot h_6$, siendo todos ellos libres de cuadrados, y como el mínimo cuerpo de característica 3 que contiene las raíces de $h_5$ es $F_{3^{9}}$, coincidiendo 9 con su grado, se tiene que es irreducible. Análogamente se tiene que para $h_6$ es $F_{3^{1}}$, y por tanto es irreducible. Mientras para $h_4$ es $F_{3^{39}}$, mayor de su grado, y por tanto aplicamos el algoritmo de nuevo a este polinomio:

\begin{verbatim}
[0] x^2i mod f = 1
[1] x^2i mod f = y^3
[2] x^2i mod f = y^6
[3] x^2i mod f = y^9
[4] x^2i mod f = y^12
[5] x^2i mod f = y^15
[6] x^2i mod f = 2*y^15 + 2*y^11 + y^10 + y^9 + 2*y^7 + y^6 + y^3 + 2*y + 2
[7] x^2i mod f = y^15 + 2*y^14 + y^13 + y^12 + y^11 + y^10 + y^7 + 2*y^4 + 
y^3 + y + 1
[8] x^2i mod f = y^14 + 2*y^13 + y^12 + y^11 + y^9 + 2*y^8 + 2*y^7 + 2*y^5 + 
2*y^2 + y + 1
[9] x^2i mod f = y^15 + y^14 + 2*y^13 + y^10 + 2*y^7 + 2*y^6 + 2*y^4 + 
y^2 + y + 1
[10] x^2i mod f = 2*y^15 + 2*y^12 + 2*y^10 + y^8 + 2*y^5 + 2*y^4 + y^3 + 
y^2
[11] x^2i mod f = 2*y^13 + 2*y^11 + 2*y^10 + 2*y^9 + 2*y^8 + y^5 + 2*y^3 + 
y + 1
[12] x^2i mod f = y^15 + y^14 + 2*y^12 + y^10 + y^8 + 2*y^7 + y^6 + 2*y^5 + 
y^4 + 2*y^2 + 2
[13] x^2i mod f = y^14 + 2*y^13 + 2*y^12 + y^10 + 2*y^9 + y^6 + y^5 + y^4 + 
2*y^2 + 1
[14] x^2i mod f = 2*y^15 + y^12 + y^11 + 2*y^10 + y^9 + 2*y^8 + 2*y^6 + y^4 + 
y^2 + y + 1
[15] x^2i mod f = 2*y^15 + y^14 + 2*y^13 + y^12 + 2*y^10 + y^9 + 2*y^7 + 2*y^6 + 
y^5 + y^4 + y + 1
\end{verbatim}

Y obtenemos la matriz:

\begin{verbatim}
[1 0 0 0 0 0 0 0 0 0 0 0 0 0 0 0]
[0 0 0 1 0 0 0 0 0 0 0 0 0 0 0 0]
[0 0 0 0 0 0 1 0 0 0 0 0 0 0 0 0]
[0 0 0 0 0 0 0 0 0 1 0 0 0 0 0 0]
[0 0 0 0 0 0 0 0 0 0 0 0 1 0 0 0]
[0 0 0 0 0 0 0 0 0 0 0 0 0 0 0 1]
[2 2 0 1 0 0 1 2 0 1 1 2 0 0 0 2]
[1 1 0 1 2 0 0 1 0 0 1 1 1 1 2 1]
[1 1 2 0 0 2 0 2 2 1 0 1 1 2 1 0]
[1 1 1 0 2 0 2 2 0 0 1 0 0 2 1 1]
[0 0 1 1 2 2 0 0 1 0 2 0 2 0 0 2]
[1 1 0 2 0 1 0 0 2 2 2 2 0 2 0 0]
[2 0 2 0 1 2 1 2 1 0 1 0 2 0 1 1]
[1 0 2 0 1 1 1 0 0 2 1 0 2 2 1 0]
[1 1 1 0 1 0 2 0 2 1 2 1 1 0 0 2]
[1 1 0 0 1 1 2 2 0 1 2 0 1 2 1 2]
\end{verbatim}

Como rango de (B-Id) es 14, tenemos que 16-14=2 soluciones que tenemos. Las soluciones son:

\begin{verbatim}
[1 0 0 0 0 0 0 0 0 0 0 0 0 0 0 0]
[0 1 0 2 0 2 2 0 1 1 1 2 2 1 2 0]
\end{verbatim}

Luego obtenemos el polinomio $f_5=y+2y^3+2y^5+2y^6+y^8+y^9+y^{10}+2y^{11}+2y^{12}+y^{13}+2y^{14}$.

Calculamos ahora los mcd asociados a $f_5$, llamando $h_{10}=(g, f_5)$, $h_{11}=(g, f_5+1)$ y $h_{12}=(g, f_5+2)$:

\begin{verbatim}
(g, f5) = y^13 + 2*y^12 + y^11 + y^10 + 2*y^9 + 2*y^8 + 2*y^7 + y^5 + y^4 + y^2 + 2
(g, f5+1) = 1
(g, f5+2) = y^3 + 2*y^2 + 1
\end{verbatim}

Obtenemos los polinomios $h_{10}, h_{11}. h_{12}$ y todos ellos son libres de cuadrados e irreducibles, pues el mínimo cuerpo de característica 3 que contiene las raíces de $h_{10}$ es $F_{3^{13}}$, a $h_{11}$ es $F_{3^{1}}$ y a $h_{12}$ es $F_{3^{3}}$. Por tanto $h_4=h_{10}\cdot h_{11}\cdot h_{12}$ es su descomposición en irreducibles y la de h queda como:

$h = (y^3 + 2y^2 + 1)\cdot (y^{13} + 2y^{12} + y^{11} + y^{10} + 2*y^9 + 2*y^8 + 2*y^7 + y^5 + y^4 + y^2 + 2)\cdot (y+2)\cdot (y^9 + 2y^7 + 2y^6 + 2y^5 + 2y^3 + y + 1)$

Finalmente, para $f_4$, considerando $h_7=(g, f_4)$, $h_8=(g, f_4+1)$ y $h_9=(g, f_4+2)$:

\begin{verbatim}
(g, f4) = y + 2
(g, f4+1) = y^22 + 2*y^21 + y^19 + y^18 + y^17 + 2*y^15 + y^14 + 2*y^13 + y^12 +
 2*y^11 + 2*y^10 + y^9 + 2*y^8 + y^7 + 2*y^6 + 2*y^5 + y^4 + 2*y^3 + y^2 + 2*y + 2
(g, f4+2) = y^3 + 2*y^2 + 1
\end{verbatim}

Obtenemos por tanto que $h=h_7\cdot h_8\cdot h_9$, siendo estos libres de cuadrados, y como el mínimo cuerpo de característica 3 que contiene las raíces de $h_7$ es $F_{3^{1}}$, coincidiendo 1 con su grado, se tiene que es irreducible. Análogamente se tiene que para $h_9$ es $F_{3^{3}}$, y por tanto es irreducible. Mientras para $h_8$ es $F_{3^{117}}$, con 117 mayor de su grado, y por tanto aplicamos el algoritmo de nuevo a este polinomio.

De nuevo, se obtiene:

\begin{verbatim}
[0] x^2i mod f = 1
[1] x^2i mod f = y^3
[2] x^2i mod f = y^6
[3] x^2i mod f = y^9
[4] x^2i mod f = y^12
[5] x^2i mod f = y^15
[6] x^2i mod f = y^18
[7] x^2i mod f = y^21
[8] x^2i mod f = y^20 + y^18 + y^15 + 2*y^14 + y^13 + y^12 + y^11 + y^10 + 2*y^9 +
 y^8 + y^7 + y^6 + y^5 + 2*y^4 + y^3 + y^2 + 2*y + 1
[9] x^2i mod f = 2*y^21 + 2*y^20 + y^19 + 2*y^18 + y^17 + 2*y^16 + y^15 + y^14 +
 y^13 + 2*y^12 + y^10 + y^9 + y^8 + 2*y^7 + y^5 + 2*y^4 + y^3 + 2*y + 1
[10] x^2i mod f = 2*y^21 + y^20 + y^18 + y^17 + 2*y^15 + y^13 + 2*y^12 + 2*y^11 +
 2*y^10 + y^8 + 2*y^6 + 2*y^4 + y^3 + y^2 + 2
[11] x^2i mod f = 2*y^21 + 2*y^20 + y^19 + 2*y^18 + 2*y^17 + 2*y^16 + y^15 + y^13 +
 2*y^12 + 2*y^11 + 2*y^10 + 2*y^8 + y^7 + 2*y^6 + y^4 + y^3 + 2*y^2
[12] x^2i mod f = 2*y^21 + 2*y^20 + y^18 + 2*y^15 + 2*y^14 + 2*y^13 + y^12 + y^10 +
 2*y^9 + 2*y^7 + 2*y^6 + 2*y^5 + y^2 + 2
[13] x^2i mod f = 2*y^19 + y^16 + y^14 + y^12 + y^10 + y^8 + 2*y^7 + y^4 + y^3 +
 2*y^2 + 2*y + 1
[14] x^2i mod f = 2*y^21 + 2*y^19 + y^18 + 2*y^17 + y^14 + y^12 + y^10 + y^9 +
 2*y^8 + 2*y^7 + y^5 + y^2 + 2*y + 2
[15] x^2i mod f = y^20 + y^19 + 2*y^17 + 2*y^15 + 2*y^14 + 2*y^13 + y^12 +
 2*y^9 + 2*y^8 + y^6 + 2*y^5 + y^4 + 1
[16] x^2i mod f = 2*y^21 + y^20 + 2*y^18 + 2*y^15 + 2*y^14 + y^13 + y^12 + 2*y^11 +
 y^10 + y^5 + 2*y^4 + 2*y^3 + 2*y^2 + 2
[17] x^2i mod f = y^20 + y^19 + 2*y^18 + y^17 + 2*y^16 + 2*y^12 + y^11 + 2*y^10 +
 y^9 + y^7 + y^5 + y^4 + y^3 + 2*y^2
[18] x^2i mod f = y^21 + 2*y^19 + y^17 + y^16 + 2*y^10 + 2*y^9 + 2*y^8 + y^6 +
 2*y^4 + y^3 + 2*y^2 + 2
[19] x^2i mod f = 2*y^21 + 2*y^20 + 2*y^19 + 2*y^18 + y^17 + 2*y^13 + y^12 + 2*y^11 +
 y^9 + y^7 + y^6 + 2*y^5 + 2*y^3 + 2*y^2 + y
[20] x^2i mod f = y^20 + 2*y^18 + 2*y^17 + y^16 + 2*y^15 + y^14 + y^13 + y^12 +
 2*y^11 + 2*y^10 + y^6 + y^3 + y
[21] x^2i mod f = y^20 + 2*y^19 + 2*y^16 + y^15 + 2*y^14 + 2*y^13 + 2*y^11 +
 y^9 + y^4 + 2*y + 1
\end{verbatim}

Y la matriz resulta:

\begin{verbatim}
[1 0 0 0 0 0 0 0 0 0 0 0 0 0 0 0 0 0 0 0 0 0]
[0 0 0 1 0 0 0 0 0 0 0 0 0 0 0 0 0 0 0 0 0 0]
[0 0 0 0 0 0 1 0 0 0 0 0 0 0 0 0 0 0 0 0 0 0]
[0 0 0 0 0 0 0 0 0 1 0 0 0 0 0 0 0 0 0 0 0 0]
[0 0 0 0 0 0 0 0 0 0 0 0 1 0 0 0 0 0 0 0 0 0]
[0 0 0 0 0 0 0 0 0 0 0 0 0 0 0 1 0 0 0 0 0 0]
[0 0 0 0 0 0 0 0 0 0 0 0 0 0 0 0 0 0 1 0 0 0]
[0 0 0 0 0 0 0 0 0 0 0 0 0 0 0 0 0 0 0 0 0 1]
[1 2 1 1 2 1 1 1 1 2 1 1 1 1 2 1 0 0 1 0 1 0]
[1 2 0 1 2 1 0 2 1 1 1 0 2 1 1 1 2 1 2 1 2 2]
[2 0 1 1 2 0 2 0 1 0 2 2 2 1 0 2 0 1 1 0 1 2]
[0 0 2 1 1 0 2 1 2 0 2 2 2 1 0 1 2 2 2 1 2 2]
[2 0 1 0 0 2 2 2 0 2 1 0 1 2 2 2 0 0 1 0 2 2]
[1 2 2 1 1 0 0 2 1 0 1 0 1 0 1 0 1 0 0 2 0 0]
[2 2 1 0 0 1 0 2 2 1 1 0 1 0 1 0 0 2 1 2 0 2]
[1 0 0 0 1 2 1 0 2 2 0 0 1 2 2 2 0 2 0 1 1 0]
[2 0 2 2 2 1 0 0 0 0 1 2 1 1 2 2 0 0 2 0 1 2]
[0 0 2 1 1 1 0 1 0 1 2 1 2 0 0 0 2 1 2 1 1 0]
[2 0 2 1 2 0 1 0 2 2 2 0 0 0 0 0 1 1 0 2 0 1]
[0 1 2 2 0 2 1 1 0 1 0 2 1 2 0 0 0 1 2 2 2 2]
[0 1 0 1 0 0 1 0 0 0 2 2 1 1 1 2 1 2 2 0 1 0]
[1 2 0 0 1 0 0 0 0 1 0 2 0 2 2 1 2 0 0 2 1 0]
\end{verbatim}

Como rango de (B-Id) se tiene 20, entonces 22-20=2 soluciones, que son:

\begin{verbatim}
[1 0 0 0 0 0 0 0 0 0 0 0 0 0 0 0 0 0 0 0 0 0]
[0 0 2 1 1 1 1 1 1 1 0 1 0 0 1 0 2 0 2 1 2 0]
\end{verbatim}

Luego obtenemos el polinomio $f_6=2*y^2+y^3+y^4+y^5+y^6+y^7+y^8+y^9+y^11+y^14+2*y^16+2*y^18+y^19+2*y^20$

Consideramos $h_{13}=(g, f_6)$, $h_{14}=(g, f_6+1)$ y $h_{15}=(g, f_6+2)$:

\begin{verbatim}
(g, f6) = 1
(g, f6+1) = y^13 + 2*y^12 + y^11 + y^10 + 2*y^9 + 2*y^8 + 2*y^7 +
 y^5 + y^4 + y^2 + 2
(g, f6+2) = y^9 + 2*y^7 + 2*y^6 + 2*y^5 + 2*y^3 + y + 1
\end{verbatim}

Obtenemos lso polinomios $h_{13}, h_{14}. h_{15}$ y todos ellos son libres de cuadrados e irreducibles, pues el mínimo cuerpo de característica 3 que contiene las raíces de $h_{13}$ es $F_{3^{1}}$, de $h_{14}$ es $F_{3^{13}}$ y de $h_{15}$ es $F_{3^{9}}$. Por tanto $h_8=h_{13}\cdot h_{14}\cdot h_{15}$ es su descomposición en irreducibles y la de h queda como:

$h = (y^3 + 2y^2 + 1)\cdot (y^{13} + 2y^{12} + y^{11} + y^{10} + 2*y^9 + 2*y^8 + 2*y^7 + y^5 + y^4 + y^2 + 2)\cdot (y+2)\cdot (y^9 + 2y^7 + 2y^6 + 2y^5 + 2y^3 + y + 1)$


\subsection{Apartado 5}
\textbf{¿ Qué deduces sobre la reducibilidad de f(x) en $\mathbb{Z}[x]$ ?}

Como $g$ descompone en $\mathbb{Z}_2[x]$ en 3 polinomios irreducibles de grados 1, 2 y 23, y en $\mathbb{Z}_3[x]$ en 4 polinomios irreducibles de grados 1, 3, 9 y 13, tenemos que las factorizaciones no son incompatibles y por tanto no podemos asegurar que $g$ sea irreducible en $\mathbb{Z}[x]$, ya que podría descomponer en un polinomio de grado 1 y otro de grado 25.

Pese a esto, lo he comprobado y el polinomio en efecto es irreducible.


\end{document}