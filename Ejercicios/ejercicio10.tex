\documentclass[a4paper]{article}
%\usepackage[spanish]{babel}
\usepackage[utf8]{inputenc}
\usepackage{amsfonts}
\usepackage{amsmath}
\usepackage{graphicx}
\usepackage{float}
%\graphicspath{ {images/} }
\usepackage{hyperref}
\usepackage{enumerate}
\title {\fbox{\Huge{\textbf{Ejercicio 10}}}}
\author {\fbox{Ana Buendía Ruiz-Azuaga}}
%\date {}
\begin{document}
\maketitle
%\tableofcontents


\section{Ejercicio 10}

\subsection{Apartado 1}

\textbf{Toma tu número p de la lista publicada para este ejercicio.}

$n = $

\textbf{Calcula el símbolo de Jacobi $\left( \frac{-11}{p} \right)$. Si sale 1, usa el algoritmo de Tonelli-Shanks para hallar soluciones a la congruencia $x^2 \equiv -11 \mod p$.}

\subsection{Apartado 2}
\textbf{Usa una de esas soluciones para factorizar el ideal principal, (p) =(p,n+$\sqrt{-11}$)(p,n+$\sqrt{-11}$) como producto de dos ideales.}



\subsection{Apartado 3}
\textbf{Aplica el algoritmo de Cornachia- Smith modificado a 2p y n para encontrar una solucíón a la ecuación diofántica $4p =x^2 + 11y^2$ y la usas para encontrar una factorización de p en a.e. del cuerpo $\mathbb{Q}(\sqrt{-11})$.}



\subsection{Apartado 4}
\textbf{¿ Son principales tus ideales (p,n+$\sqrt{-11}$)(p,n+$\sqrt{-11}$)}



\end{document}