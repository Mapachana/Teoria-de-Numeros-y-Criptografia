\documentclass[a4paper]{article}
%\usepackage[spanish]{babel}
\usepackage[utf8]{inputenc}
\usepackage{amsfonts}
\usepackage{amsmath}
\usepackage{graphicx}
\usepackage{float}
%\graphicspath{ {images/} }
\usepackage{hyperref}
\usepackage{enumerate}
\title {\fbox{\Huge{\textbf{Ejercicio 10}}}}
\author {\fbox{Ana Buendía Ruiz-Azuaga}}
%\date {}
\begin{document}
\maketitle
%\tableofcontents


\section{Ejercicio 10}

\subsection{Apartado 1}

\textbf{Toma tu número p de la lista publicada para este ejercicio.}

$p = 77770081$

\textbf{Calcula el símbolo de Jacobi $\left( \frac{-11}{p} \right)$. Si sale 1, usa el algoritmo de Tonelli-Shanks para hallar soluciones a la congruencia $x^2 \equiv -11 \mod p$.}

Tenemos que $\left( \frac{-11}{p}\right)=1$ luego aplicamos el algoritmo de Tonelli-Shanks para hallar las soluciones de $x^2\equiv -11\mod p$.

Para ello, comenzamos factorizando $p-1=2^5\cdot 3^2\cdot 5 \cdot 53\cdot 1019$

Para ello vamos a usar el método $\rho$ de Polard. Comenzamos sacando los factores 2, y aplicamos el algoritmo a 2430315. Para factorizar este número han sido necesarias un total de 11 iteraciones.

Finalmente, aplicamos Lucas-Lehmer, obtenemos que 17 es elemento primitivo para $p$, ya que $17^{p-1}\equiv 1\mod p$ y $17^{\frac{p-1}{p}}\not\equiv 1\mod p $ para $p\in\{2,3,5, 53, 1019\}$ pues::

\begin{verbatim}
17^(p-1)/ 2 = 77770080 mod p
17^(p-1)/ 3 = 58134188 mod p
17^(p-1)/ 5 = 66432901 mod p
17^(p-1)/ 53 = 68065795 mod p
17^(p-1)/ 1019 = 65224721 mod p
\end{verbatim}

Luego $p$ es primo

Como $p\equiv 1\mod 8$ tenemos que usar Tonelli-Shanks.

Hemos visto además que $p-1=2^5\cdot 2430315$, y  como $(-11)^{2430315} \equiv 158982 \mod p \not\equiv 1\mod p$.
Y este tiene orden $2^2 \mod p$.

Calculamos ahora un no residuo cuadrático módulo p, y el primero es $n=17$.

Por tanto un generador del 2-subgrupo de Sylow $G=\mathbb{Z}_{2^{17}}$.

Aplicando el algoritmo obtenemos:

\begin{verbatim}
z: 55328379, t: 158982, i: 2, r: 29971170
b: 12935680, t1: 1, i1: 0, r1: 796425
Soluciones: 796425 76973656
\end{verbatim}

Luego la soluciones son $796425$ y $76973656$, que se corresponden con $r_1$ y $p-r_1$.

\subsection{Apartado 2}
\textbf{Usa una de esas soluciones para factorizar el ideal principal, (p) =(p,n+$\sqrt{-11}$)(p,n+$\sqrt{-11}$) como producto de dos ideales.}

Tomamos la solución impar $n=796425$ y como $p$ es un impar primo que no divide a -11 entonces $(p)=\left(p, 796425+\sqrt{-11}\right)\left(796425-\sqrt{-11}\right)$

\subsection{Apartado 3}
\textbf{Aplica el algoritmo de Cornachia- Smith modificado a 2p y n para encontrar una solucíón a la ecuación diofántica $4p =x^2 + 11y^2$ y la usas para encontrar una factorización de p en a.e. del cuerpo $\mathbb{Q}(\sqrt{-11})$.}

Aplicamos el algoritmo de Cornachia-Smith:

\begin{verbatim}
Paso 1:   155540162 = 195*796425 + 237287
Paso 2:   796425 = 3*237287 + 84564
Paso 3:   237287 = 2*84564 + 68159
Paso 4:   84564 = 1*68159 + 16405
\end{verbatim}

Tras 4 divisiones obtenemos el resto $x=16405$. Por tanto, podemos hallar $y$ despejando de la ecuación:
$$4p =x^2 + 11y^2  \Leftrightarrow y = \sqrt{\frac{4p-x^2}{11}}=1953$$

Así, se cumple  $4p = 16405^2+11\cdot 1953^2$

Luego tenemos que la factorización de $p$ en a.e. de $\mathbb{Q}[\sqrt{p}]$ es
$$p= \left( \frac{x+y\sqrt{-11}}{2}\right) \left( \frac{x-y\sqrt{-11}}{2}\right) = \left( \frac{16405+1953\sqrt{-11}}{2}\right) \left( \frac{16405-1953\sqrt{-11}}{2}\right)$$


\subsection{Apartado 4}
\textbf{¿ Son principales tus ideales (p,n+$\sqrt{-11}$)(p,n+$\sqrt{-11}$)}

Son principales ya que $\mathbb{Q}[\sqrt{-11}]$ es DIP.


\end{document}