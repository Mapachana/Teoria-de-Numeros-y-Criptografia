\documentclass[a4paper]{article}
%\usepackage[spanish]{babel}
\usepackage[utf8]{inputenc}
\usepackage{amsfonts}
\usepackage{amsmath}
\usepackage{graphicx}
\usepackage{float}
%\graphicspath{ {images/} }
\usepackage{hyperref}
\usepackage{enumerate}
\title {\fbox{\Huge{\textbf{Ejercicio 3}}}}
\author {\fbox{Ana Buendía Ruiz-Azuaga}}
%\date {}
\begin{document}
\maketitle
%\tableofcontents


\section{Ejercicio 3}
\subsection{Apartado 1}
\textbf{Dado tu número m (de 30 cifras o mas) de la lista publicada.}

$$m=36580545945776718558633000960211$$

\textbf{Calcula $a^{m-1} \mod m$, para los 5 primeros primos.}

Aplicando el algoritmo de exponenciación rápida del primer ejercicio obtenemos que:

$$2^{m-1}\equiv 1\mod m$$
$$3^{m-1}\equiv 1\mod m$$
$$5^{m-1}\equiv 1\mod m$$
$$7^{m-1}\equiv 1\mod m$$
$$11^{m-1}\equiv 1\mod m$$


\subsection{Apartado 2}
\textbf{Calcula el test de Solovay-Strassen para los 5 primeros primos.}

Del primer apartado tenemos que el número es posible primo de Fermat, pues $a^{m-1}\equiv 1\mod m$ para $a$ siendo los 5 primeros primos.

Comprobamos ahora si es posible primo de Euler, es decir, si cumple $\left( \frac{p}{m} \right)= p^{\frac{m-1}{2}}\mod m$ para los 5 primeros primos.

Para $p=2$ tenemos que como $m\equiv 3 \mod 8$ entonces $\left( \frac{2}{m}\right)=(-1)^{\frac{m^2-1}{8}}=-1$.

Además, $2^{\frac{m-1}{2}}\equiv -1\mod m$, que coincide con su símbolo de Jacobi.

Para $p=3$ como $m\equiv 3\mod 4$ entonces $\left( \frac{3}{m}\right)=-\left( \frac{m}{3}\right)$ y como $m\equiv 1\mod 3$ entonces $-\left( \frac{m}{3}\right)=-\left( \frac{1}{3}\right)=-1$.

Además, $3^{\frac{m-1}{2}}\equiv -1\mod m$, que coincide con su símbolo de Jacobi.

Para $p=5$ como $5\equiv 1\mod 4$ entonces $\left( \frac{5}{m}\right)=\left( \frac{m}{5}\right)$ y como $m\equiv 1\mod 5$ entonces $\left( \frac{m}{5}\right)=\left( \frac{1}{5}\right)=1$.

Además, $5^{\frac{m-1}{2}}\equiv 1\mod m$, que coincide con su símbolo de Jacobi.

Para $p=7$ como $m\equiv 3\mod 4$ entonces $\left( \frac{7}{m}\right)=-\left( \frac{m}{7}\right)$ y como $m\equiv 5\mod 7$ entonces $-\left( \frac{m}{7}\right)=-\left( \frac{5}{7}\right)=1$.

Además, $7^{\frac{m-1}{2}}\equiv 1\mod m$, que coincide con su símbolo de Jacobi.

Para $p=11$ como $m\equiv 3\mod 4$ entonces $\left( \frac{11}{m}\right)=-\left( \frac{m}{11}\right)$ y como $m\equiv 6\mod 11$ entonces $-\left( \frac{m}{11}\right)=-\left( \frac{6}{11}\right)=1$.

Además, $11^{\frac{m-1}{2}}\equiv 1\mod m$, que coincide con su símbolo de Jacobi.

Luego $m$ es posible primo de Euler para todas las bases probadas.

Y, como es posible primo de Fermat y posible primo de Euler para los 5 primeros primo, se tiene que pasa el test de Solovay-Strassen para los 5 primeros primos.

\subsection{Apartado 3}

\textbf{Calcula el test de Miller-Rabin para esas 5 bases.}

Para comprobar el test de Miller-Rabin vamos a construir la a-sucesión correspondiente. Comenzamos descomponiendo $m-1$ como $m-1=2^rn$.

Como obtenemos que $r=1$ tenemos que toda a-sucesión va a tener 2 términos.

La a-sucesion obtenida para la base 2 es:
\begin{verbatim}
[36580545945776718558633000960210, 1]
\end{verbatim}
que sería: $2^{\frac{m-1}{2}}\equiv -1\mod m$, $2^{m-1}\equiv 1\mod m$.

La a-sucesion obtenida para la base 3 es:
\begin{verbatim}
[36580545945776718558633000960210, 1]
\end{verbatim}
que sería: $3^{\frac{m-1}{2}}\equiv -1\mod m$, $3^{m-1}\equiv 1\mod m$.

La a-sucesion obtenida para la base 5 es:
\begin{verbatim}
[1, 1]
\end{verbatim}
que sería: $5^{\frac{m-1}{2}}\equiv 1\mod m$, $5^{m-1}\equiv 1\mod m$.


La a-sucesion obtenida para la base 7 es:
\begin{verbatim}
[1, 1]
\end{verbatim}
que sería: $7^{\frac{m-1}{2}}\equiv 1\mod m$, $7^{m-1}\equiv 1\mod m$.

La a-sucesion obtenida para la base 11 es:
\begin{verbatim}
[1, 1]
\end{verbatim}
que sería: $11^{\frac{m-1}{2}}\equiv 1\mod m$, $11^{m-1}\equiv 1\mod m$.

Teniendo en cuenta que $36580545945776718558633000960210\equiv -1\mod m$ tenemos que $m$ pasa el test de Miller-Rabin para los 5 primeros primos, pues las sucesiones acaban en 1 y todo 1 va precedido de otro 1 o de -1.


\subsection{Apartado 4}
\textbf{¿Qué deduces sobre la primalidad de tu número?}

En el apartado 2 hemos comprobado que el número pasa el test de Solovay-Strassen para los 5 primeros primos, luego la probabilidad de que el número sea primo es muy alta.

Además, dado que ha pasado el test de Miller-Rabin para los 5 primeros primos, la probabilidad de que sea primo es mayor de $1-\frac{1}{4^5}$.

Luego la probabilidad de que el número sea primo es muy alta.


\end{document}