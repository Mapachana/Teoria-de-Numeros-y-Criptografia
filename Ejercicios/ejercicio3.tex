\documentclass[a4paper]{article}
%\usepackage[spanish]{babel}
\usepackage[utf8]{inputenc}
\usepackage{amsfonts}
\usepackage{amsmath}
\usepackage{graphicx}
\usepackage{float}
%\graphicspath{ {images/} }
\usepackage{hyperref}
\usepackage{enumerate}
\title {\fbox{\Huge{\textbf{Ejercicio 3}}}}
\author {\fbox{Ana Buendía Ruiz-Azuaga}}
%\date {}
\begin{document}
\maketitle
%\tableofcontents


\section{Ejercicio 3}
\subsection{Apartado 1}
\textbf{Dado tu número m (de 30 cifras o mas) de la lista publicada.}

$$m=36580545945776718558633000960211$$

\textbf{Calcula $a^{m-1} \mod m$, para los 5 primeros primos.}



\subsection{Apartado 2}
\textbf{Calcula el test de Solovay-Strassen para los 5 primeros primos.}


\subsection{Apartado 3}

\textbf{Calcula el test de Miller-Rabin para esas 5 bases.}



\subsection{Apartado 4}
\textbf{¿Qué deduces sobre la primalidad de tu número?}



\end{document}