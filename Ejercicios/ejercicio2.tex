\documentclass[a4paper]{article}
%\usepackage[spanish]{babel}
\usepackage[utf8]{inputenc}
\usepackage{amsfonts}
\usepackage{amsmath}
\usepackage{graphicx}
\usepackage{float}
%\graphicspath{ {images/} }
\usepackage{hyperref}
\usepackage{enumerate}
\title {\fbox{\Huge{\textbf{Ejercicio 2}}}}
\author {\fbox{Ana Buendía Ruiz-Azuaga}}
%\date {}
\begin{document}
\maketitle
%\tableofcontents


\section{Ejercicio 2}
\subsection{Apartado 1}
\textbf{Dado tu número n = 77770081 de 8 cifras de la lista publicada.}

\textbf{Usa el algoritmo manual para calcular el símbolo de Jacobi$\left(\frac{p}{n}\right)$,para p cada uno de los 5 primeros primos.}

Comenzamos con $p=2$:

Tenemos que $n \mod 8\equiv 1 \mod 8$, luego $\left( \frac{2}{n}\right)=(-1)^{\frac{n^2-1}{8}}=1$

Ahora consideramos $p=3$:

Como $n\equiv 1 \mod 4$ entonces $\left( \frac{p}{n} \right)=\left( \frac{n}{p} \right)$.

Además, como $n\equiv 1 \mod 3$ entonces $\left( \frac{n}{3} \right)=\left( \frac{1}{3} \right)$.

Y, como 1 es un cuadrado módulo $3$ se tiene que $\left( \frac{1}{3} \right)=1$

Ahora tomamos $p=5$:

Como $n\equiv 1 \mod 4$ tenemos $\left( \frac{p}{n} \right) = \left( \frac{n}{p} \right)$.

Y como $n \equiv 1 \mod 5$ tenemos $\left( \frac{n}{5} \right)=\left( \frac{1}{5} \right)$.

De donde, usando que $1$ es cuadrado módulo $5$,se tiene que $\left( \frac{1}{5} \right)=1$.

Tomamos ahora $p=7$:

Como $n\equiv 1 \mod 4$ tenemos $\left( \frac{p}{n} \right) = \left( \frac{n}{p} \right)$.

Y como $n \equiv 4 \mod 7$ tenemos $\left( \frac{n}{7} \right)=\left( \frac{4}{7} \right)$.

De donde se tiene que $\left( \frac{4}{7} \right)=4^{\frac{7-1}{2}}\mod 7=1$.

Finalmente consideramos $p=11$:

Como $n\equiv 1 \mod 4$ tenemos $\left( \frac{p}{n} \right) = \left( \frac{n}{p} \right)$.

Y como $n \equiv 4 \mod 11$ tenemos $\left( \frac{n}{11} \right)=\left( \frac{4}{11} \right)$.

De donde se tiene que $\left( \frac{4}{11} \right)=4^{\frac{11-1}{2}}\mod 11=1$.



\subsection{Apartado 2}
\textbf{Si para alguna de esas bases tu número sale posible primo de Fermat, comprueba si además es posible primo de Euler}

Comprobamos si $n$ es un posible primo de Fermat para estas bases. Para ello, comprobamos si se cumple $a^{n-1}\equiv 1 \mod n$.

Para esta comprobación usamos el algoritmo de exponenciación rápida de derecha a izquierda del ejercicio anterior:

\subsubsection{Posibles primos de Fermat}

El resultado del algoritmo es el último valor de la variable acu.

\textbf{Base 2}

\begin{verbatim}
Paso: 0,  acu: 1,  base: 2  
Paso: 1,  acu: 1,  base: 4  
Paso: 2,  acu: 1,  base: 16  
Paso: 3,  acu: 1,  base: 256  
Paso: 4,  acu: 1,  base: 65536  
Paso: 5,  acu: 1,  base: 17612841  
Paso: 6,  acu: 17612841,  base: 69275565  
Paso: 7,  acu: 44516305,  base: 9980674  
Paso: 8,  acu: 44516305,  base: 23223320  
Paso: 9,  acu: 44680160,  base: 67690927  
Paso: 10,  acu: 44680160,  base: 77257279  
Paso: 11,  acu: 1465333,  base: 25247343  
Paso: 12,  acu: 70708033,  base: 13797891  
Paso: 13,  acu: 70708033,  base: 15599233  
Paso: 14,  acu: 31848612,  base: 19426093  
Paso: 15,  acu: 31848612,  base: 70568710  
Paso: 16,  acu: 19282073,  base: 10086087  
Paso: 17,  acu: 19282073,  base: 52267494  
Paso: 18,  acu: 44352822,  base: 10929208  
Paso: 19,  acu: 44352822,  base: 53478878  
Paso: 20,  acu: 44352822,  base: 18488420  
Paso: 21,  acu: 44352822,  base: 3628315  
Paso: 22,  acu: 35935437,  base: 61507869  
Paso: 23,  acu: 35935437,  base: 45590014  
Paso: 24,  acu: 15507894,  base: 21699431  
Paso: 25,  acu: 15507894,  base: 50932700  
Paso: 26,  acu: 15507894,  base: 44851232  
Paso: 27,  acu: 1,  base: 55498452  
\end{verbatim}

Luego $2^{n-1}\equiv 1 \mod n$.


\textbf{Base 3}
\begin{verbatim}
Paso: 0,  acu: 1,  base: 3  
Paso: 1,  acu: 1,  base: 9  
Paso: 2,  acu: 1,  base: 81  
Paso: 3,  acu: 1,  base: 6561  
Paso: 4,  acu: 1,  base: 43046721  
Paso: 5,  acu: 1,  base: 12562698  
Paso: 6,  acu: 12562698,  base: 67023312  
Paso: 7,  acu: 58342833,  base: 40759744  
Paso: 8,  acu: 58342833,  base: 63810002  
Paso: 9,  acu: 24303074,  base: 70788665  
Paso: 10,  acu: 24303074,  base: 26430655  
Paso: 11,  acu: 58499677,  base: 49886400  
Paso: 12,  acu: 54301195,  base: 47309308  
Paso: 13,  acu: 54301195,  base: 31523377  
Paso: 14,  acu: 73546473,  base: 66869943  
Paso: 15,  acu: 73546473,  base: 481537  
Paso: 16,  acu: 19632816,  base: 45270908  
Paso: 17,  acu: 19632816,  base: 75692200  
Paso: 18,  acu: 29496059,  base: 27863284  
Paso: 19,  acu: 29496059,  base: 52818504  
Paso: 20,  acu: 29496059,  base: 43958962  
Paso: 21,  acu: 29496059,  base: 41181807  
Paso: 22,  acu: 3749083,  base: 50088853  
Paso: 23,  acu: 3749083,  base: 51464019  
Paso: 24,  acu: 51758518,  base: 62708747  
Paso: 25,  acu: 51758518,  base: 43324625  
Paso: 26,  acu: 51758518,  base: 75071723  
Paso: 27,  acu: 1,  base: 67602701  
\end{verbatim}

Luego $3^{n-1}\equiv 1 \mod n$.


\textbf{Base 5}
\begin{verbatim}
Paso: 0,  acu: 1,  base: 5  
Paso: 1,  acu: 1,  base: 25  
Paso: 2,  acu: 1,  base: 625  
Paso: 3,  acu: 1,  base: 390625  
Paso: 4,  acu: 1,  base: 2991703  
Paso: 5,  acu: 1,  base: 39298243  
Paso: 6,  acu: 39298243,  base: 11535691  
Paso: 7,  acu: 9210897,  base: 14558624  
Paso: 8,  acu: 9210897,  base: 42797110  
Paso: 9,  acu: 29856166,  base: 49630482  
Paso: 10,  acu: 29856166,  base: 32027512  
Paso: 11,  acu: 28016975,  base: 53957117  
Paso: 12,  acu: 65591934,  base: 52967279  
Paso: 13,  acu: 65591934,  base: 70302898  
Paso: 14,  acu: 52076763,  base: 6980919  
Paso: 15,  acu: 52076763,  base: 8687369  
Paso: 16,  acu: 30787191,  base: 38207412  
Paso: 17,  acu: 30787191,  base: 6383268  
Paso: 18,  acu: 73995456,  base: 31821494  
Paso: 19,  acu: 73995456,  base: 40939349  
Paso: 20,  acu: 73995456,  base: 48295268  
Paso: 21,  acu: 73995456,  base: 3819153  
Paso: 22,  acu: 58442021,  base: 73175778  
Paso: 23,  acu: 58442021,  base: 42371599  
Paso: 24,  acu: 63235719,  base: 62680535  
Paso: 25,  acu: 63235719,  base: 10805207  
Paso: 26,  acu: 63235719,  base: 8671437  
Paso: 27,  acu: 1,  base: 28118256  
\end{verbatim}

Luego $5^{n-1}\equiv 1 \mod n$.


\textbf{Base 7}
\begin{verbatim}
Paso: 0,  acu: 1,  base: 7  
Paso: 1,  acu: 1,  base: 49  
Paso: 2,  acu: 1,  base: 2401  
Paso: 3,  acu: 1,  base: 5764801  
Paso: 4,  acu: 1,  base: 64016519  
Paso: 5,  acu: 1,  base: 66361301  
Paso: 6,  acu: 66361301,  base: 53942426  
Paso: 7,  acu: 11146291,  base: 10572008  
Paso: 8,  acu: 11146291,  base: 3472833  
Paso: 9,  acu: 4865544,  base: 62654490  
Paso: 10,  acu: 4865544,  base: 59229057  
Paso: 11,  acu: 37101810,  base: 71051765  
Paso: 12,  acu: 58332031,  base: 36885562  
Paso: 13,  acu: 58332031,  base: 46045556  
Paso: 14,  acu: 55512383,  base: 3850811  
Paso: 15,  acu: 55512383,  base: 12933127  
Paso: 16,  acu: 42505399,  base: 61345597  
Paso: 17,  acu: 42505399,  base: 28288883  
Paso: 18,  acu: 50620210,  base: 1900642  
Paso: 19,  acu: 50620210,  base: 19749714  
Paso: 20,  acu: 50620210,  base: 28031156  
Paso: 21,  acu: 50620210,  base: 48437372  
Paso: 22,  acu: 26226364,  base: 3766720  
Paso: 23,  acu: 26226364,  base: 39291003  
Paso: 24,  acu: 40666288,  base: 69077328  
Paso: 25,  acu: 40666288,  base: 55376817  
Paso: 26,  acu: 40666288,  base: 56629098  
Paso: 27,  acu: 1,  base: 6352934  
\end{verbatim}

Luego $7^{n-1}\equiv 1 \mod n$.


\textbf{Base 11}
\begin{verbatim}
Paso: 0,  acu: 1,  base: 11  
Paso: 1,  acu: 1,  base: 121  
Paso: 2,  acu: 1,  base: 14641  
Paso: 3,  acu: 1,  base: 58818719  
Paso: 4,  acu: 1,  base: 66544732  
Paso: 5,  acu: 1,  base: 8800012  
Paso: 6,  acu: 8800012,  base: 30883746  
Paso: 7,  acu: 66090327,  base: 52473686  
Paso: 8,  acu: 66090327,  base: 52822068  
Paso: 9,  acu: 48240432,  base: 61995449  
Paso: 10,  acu: 48240432,  base: 30811749  
Paso: 11,  acu: 41802059,  base: 96486  
Paso: 12,  acu: 1523852,  base: 54908557  
Paso: 13,  acu: 1523852,  base: 38676855  
Paso: 14,  acu: 58039934,  base: 3824773  
Paso: 15,  acu: 58039934,  base: 25185105  
Paso: 16,  acu: 42622747,  base: 77338508  
Paso: 17,  acu: 42622747,  base: 73680415  
Paso: 18,  acu: 49309845,  base: 56601615  
Paso: 19,  acu: 49309845,  base: 56458770  
Paso: 20,  acu: 49309845,  base: 69634310  
Paso: 21,  acu: 49309845,  base: 31664693  
Paso: 22,  acu: 53354819,  base: 69239724  
Paso: 23,  acu: 53354819,  base: 14398341  
Paso: 24,  acu: 7518936,  base: 52010095  
Paso: 25,  acu: 7518936,  base: 7393889  
Paso: 26,  acu: 7518936,  base: 27324237  
Paso: 27,  acu: 1,  base: 74340218  
\end{verbatim}

Luego $11^{n-1}\equiv 1 \mod n$.


Como podemos ver, el número es posible primo de Fermat para todas las bases, por lo que se va a comprobar para todas ellas si es posible primo de Euler. Para ello comprobamos $\left( \frac{p}{n} \right)= p^{\frac{n-1}{2}}\mod n$ usando de nuevo el algoritmo de exponenciación rápida de derecha a izquierda del ejercicio 1.

\subsubsection{Posibles primos de Euler}

\textbf{Base 2}
\begin{verbatim}
Paso: 0,  acu: 1,  base: 2  
Paso: 1,  acu: 1,  base: 4  
Paso: 2,  acu: 1,  base: 16  
Paso: 3,  acu: 1,  base: 256  
Paso: 4,  acu: 1,  base: 65536  
Paso: 5,  acu: 65536,  base: 17612841  
Paso: 6,  acu: 11605574,  base: 69275565  
Paso: 7,  acu: 11605574,  base: 9980674  
Paso: 8,  acu: 69874828,  base: 23223320  
Paso: 9,  acu: 69874828,  base: 67690927  
Paso: 10,  acu: 13173522,  base: 77257279  
Paso: 11,  acu: 11887340,  base: 25247343  
Paso: 12,  acu: 11887340,  base: 13797891  
Paso: 13,  acu: 9967700,  base: 15599233  
Paso: 14,  acu: 9967700,  base: 19426093  
Paso: 15,  acu: 41890761,  base: 70568710  
Paso: 16,  acu: 41890761,  base: 10086087  
Paso: 17,  acu: 53220709,  base: 52267494  
Paso: 18,  acu: 53220709,  base: 10929208  
Paso: 19,  acu: 53220709,  base: 53478878  
Paso: 20,  acu: 53220709,  base: 18488420  
Paso: 21,  acu: 2277206,  base: 3628315  
Paso: 22,  acu: 2277206,  base: 61507869  
Paso: 23,  acu: 72660827,  base: 45590014  
Paso: 24,  acu: 72660827,  base: 21699431  
Paso: 25,  acu: 72660827,  base: 50932700  
Paso: 26,  acu: 1,  base: 44851232  
\end{verbatim}

Luego $2^{\frac{n-1}{2}}\equiv 1\mod n$

Y, del apartado anterior habíamos obtenido que el símbolo de Jacobi con $p=2$ era $1$, por tanto coincide y es un posible primo de Euler para la base $2$.

\textbf{Base 3}
\begin{verbatim}
Paso: 0,  acu: 1,  base: 3  
Paso: 1,  acu: 1,  base: 9  
Paso: 2,  acu: 1,  base: 81  
Paso: 3,  acu: 1,  base: 6561  
Paso: 4,  acu: 1,  base: 43046721  
Paso: 5,  acu: 43046721,  base: 12562698  
Paso: 6,  acu: 65100767,  base: 67023312  
Paso: 7,  acu: 65100767,  base: 40759744  
Paso: 8,  acu: 8749244,  base: 63810002  
Paso: 9,  acu: 8749244,  base: 70788665  
Paso: 10,  acu: 64969516,  base: 26430655  
Paso: 11,  acu: 65638599,  base: 49886400  
Paso: 12,  acu: 65638599,  base: 47309308  
Paso: 13,  acu: 58223961,  base: 31523377  
Paso: 14,  acu: 58223961,  base: 66869943  
Paso: 15,  acu: 68880686,  base: 481537  
Paso: 16,  acu: 68880686,  base: 45270908  
Paso: 17,  acu: 22173965,  base: 75692200  
Paso: 18,  acu: 22173965,  base: 27863284  
Paso: 19,  acu: 22173965,  base: 52818504  
Paso: 20,  acu: 22173965,  base: 43958962  
Paso: 21,  acu: 31467141,  base: 41181807  
Paso: 22,  acu: 31467141,  base: 50088853  
Paso: 23,  acu: 66936124,  base: 51464019  
Paso: 24,  acu: 66936124,  base: 62708747  
Paso: 25,  acu: 66936124,  base: 43324625  
Paso: 26,  acu: 1,  base: 75071723 
\end{verbatim}

Luego $3^{\frac{n-1}{2}}\equiv 1\mod n$

Y, del apartado anterior habíamos obtenido que el símbolo de Jacobi con $p=3$ era $1$, por tanto coincide y es un posible primo de Euler para la base $3$.

\textbf{Base 5}
\begin{verbatim}
Paso: 0,  acu: 1,  base: 5  
Paso: 1,  acu: 1,  base: 25  
Paso: 2,  acu: 1,  base: 625  
Paso: 3,  acu: 1,  base: 390625  
Paso: 4,  acu: 1,  base: 2991703  
Paso: 5,  acu: 2991703,  base: 39298243  
Paso: 6,  acu: 62606403,  base: 11535691  
Paso: 7,  acu: 62606403,  base: 14558624  
Paso: 8,  acu: 65051902,  base: 42797110  
Paso: 9,  acu: 65051902,  base: 49630482  
Paso: 10,  acu: 76302315,  base: 32027512  
Paso: 11,  acu: 9983068,  base: 53957117  
Paso: 12,  acu: 9983068,  base: 52967279  
Paso: 13,  acu: 57895152,  base: 70302898  
Paso: 14,  acu: 57895152,  base: 6980919  
Paso: 15,  acu: 54677894,  base: 8687369  
Paso: 16,  acu: 54677894,  base: 38207412  
Paso: 17,  acu: 465722,  base: 6383268  
Paso: 18,  acu: 465722,  base: 31821494  
Paso: 19,  acu: 465722,  base: 40939349  
Paso: 20,  acu: 465722,  base: 48295268  
Paso: 21,  acu: 50367243,  base: 3819153  
Paso: 22,  acu: 50367243,  base: 73175778  
Paso: 23,  acu: 11756279,  base: 42371599  
Paso: 24,  acu: 11756279,  base: 62680535  
Paso: 25,  acu: 11756279,  base: 10805207  
Paso: 26,  acu: 1,  base: 8671437  
\end{verbatim}

Luego $5^{\frac{n-1}{2}}\equiv 1\mod n$

De nuevo, del apartado anterior habíamos obtenido que el símbolo de Jacobi con $p=5$ era $1$, por tanto coincide y es un posible primo de Euler para la base $5$.

\textbf{Base 7}
\begin{verbatim}
Paso: 0,  acu: 1,  base: 7  
Paso: 1,  acu: 1,  base: 49  
Paso: 2,  acu: 1,  base: 2401  
Paso: 3,  acu: 1,  base: 5764801  
Paso: 4,  acu: 1,  base: 64016519  
Paso: 5,  acu: 64016519,  base: 66361301  
Paso: 6,  acu: 36776249,  base: 53942426  
Paso: 7,  acu: 36776249,  base: 10572008  
Paso: 8,  acu: 32971776,  base: 3472833  
Paso: 9,  acu: 32971776,  base: 62654490  
Paso: 10,  acu: 50357183,  base: 59229057  
Paso: 11,  acu: 23925373,  base: 71051765  
Paso: 12,  acu: 23925373,  base: 36885562  
Paso: 13,  acu: 13272104,  base: 46045556  
Paso: 14,  acu: 13272104,  base: 3850811  
Paso: 15,  acu: 44405412,  base: 12933127  
Paso: 16,  acu: 44405412,  base: 61345597  
Paso: 17,  acu: 6569097,  base: 28288883  
Paso: 18,  acu: 6569097,  base: 1900642  
Paso: 19,  acu: 6569097,  base: 19749714  
Paso: 20,  acu: 6569097,  base: 28031156  
Paso: 21,  acu: 51199192,  base: 48437372  
Paso: 22,  acu: 51199192,  base: 3766720  
Paso: 23,  acu: 17947736,  base: 39291003  
Paso: 24,  acu: 17947736,  base: 69077328  
Paso: 25,  acu: 17947736,  base: 55376817  
Paso: 26,  acu: 1,  base: 56629098  
\end{verbatim}

Luego $7^{\frac{n-1}{2}}\equiv 1\mod n$

Del apartado anterior habíamos obtenido que el símbolo de Jacobi con $p=7$ era $1$, por tanto coincide y es un posible primo de Euler para la base $7$.

\textbf{Base 11}
\begin{verbatim}
Paso: 0,  acu: 1,  base: 11  
Paso: 1,  acu: 1,  base: 121  
Paso: 2,  acu: 1,  base: 14641  
Paso: 3,  acu: 1,  base: 58818719  
Paso: 4,  acu: 1,  base: 66544732  
Paso: 5,  acu: 66544732,  base: 8800012  
Paso: 6,  acu: 39901688,  base: 30883746  
Paso: 7,  acu: 39901688,  base: 52473686  
Paso: 8,  acu: 76924925,  base: 52822068  
Paso: 9,  acu: 76924925,  base: 61995449  
Paso: 10,  acu: 55436924,  base: 30811749  
Paso: 11,  acu: 47440987,  base: 96486  
Paso: 12,  acu: 47440987,  base: 54908557  
Paso: 13,  acu: 43113226,  base: 38676855  
Paso: 14,  acu: 43113226,  base: 3824773  
Paso: 15,  acu: 66900968,  base: 25185105  
Paso: 16,  acu: 66900968,  base: 77338508  
Paso: 17,  acu: 35499153,  base: 73680415  
Paso: 18,  acu: 35499153,  base: 56601615  
Paso: 19,  acu: 35499153,  base: 56458770  
Paso: 20,  acu: 35499153,  base: 69634310  
Paso: 21,  acu: 59365955,  base: 31664693  
Paso: 22,  acu: 59365955,  base: 69239724  
Paso: 23,  acu: 2469011,  base: 14398341  
Paso: 24,  acu: 2469011,  base: 52010095  
Paso: 25,  acu: 2469011,  base: 7393889  
Paso: 26,  acu: 1,  base: 27324237 
\end{verbatim}

Luego $11^{\frac{n-1}{2}\equiv 1}\mod n$

Y, del apartado anterior habíamos obtenido que el símbolo de Jacobi con $p=11$ era $1$, por tanto coincide y es un posible primo de Euler para la base $11$.

Como todos los números obtenidos coinciden con su símbolo de Jacobi, $n$ es posible primo de Euler para todas las bases.

\subsection{Apartado 3}

\textbf{¿Es tu número $n$ pseudoprimo de Fermat o de Euler para alguna de las bases?}

Del apartado anterior tenemos que $n$ es posible primo de Fermat para todas las bases, además, también es posible primo de Euler para todas ellas. Dado que no he encontrado ningún factor suyo creo que $n$ en efecto es primo, y por tanto no sería pseudoprimo para ninguna base.

\end{document}