\documentclass[a4paper]{article}
%\usepackage[spanish]{babel}
\usepackage[utf8]{inputenc}
\usepackage{amsfonts}
\usepackage{amsmath}
\usepackage{graphicx}
\usepackage{float}
%\graphicspath{ {images/} }
\usepackage{hyperref}
\usepackage{enumerate}
\title {\fbox{\Huge{\textbf{Ejercicio 7}}}}
\author {\fbox{Ana Buendía Ruiz-Azuaga}}
%\date {}
\begin{document}
\maketitle
%\tableofcontents


\section{Ejercicio 7}

\subsection{Apartado 1}

\textbf{Toma tu número n de la lista publicada para el ejercicio 3. Sea d elprimer elemento de la sucesión 5, -7, 9, -11, 13, ... que satisface que el símbolo deJacobi es (d|n) = -1}

$n = 36580545945776718558633000960211$

\textbf{Con P = 1, Q = (1-d)/4, define el e.c. $\alpha$ y sus sucesiones de Lucas asociadas.}

\subsection{Apartado 2}
\textbf{Si n primo ¿Que debería de pasarle a $V_r,U_r$,módulo n? ¿Y a $V_{r/2} ,U_{r/2}$? Calcula los términos $V_{r/2} ,U_{r/2},V_r ,U_r$, módulo n, de las sucesiones de Lucas. ¿ Tu n verifica el TPF para el entero cuadrático $\alpha$ ?}


\subsection{Apartado 3}

\textbf{Factoriza r = n+1 y para cada factor primo p suyo, calcula $U_{r/p}$ . ¿ Cuál es el rango de Lucas w(n) ?. ¿ Qué deduces sobre la primalidad de tu n ?}

\end{document}