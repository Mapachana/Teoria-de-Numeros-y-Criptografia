\documentclass[a4paper]{article}
%\usepackage[spanish]{babel}
\usepackage[utf8]{inputenc}
\usepackage{amsfonts}
\usepackage{amsmath}
\usepackage{graphicx}
\usepackage{float}
%\graphicspath{ {images/} }
\usepackage{hyperref}
\usepackage{enumerate}
\title {\fbox{\Huge{\textbf{Ejercicio 4}}}}
\author {\fbox{Ana Buendía Ruiz-Azuaga}}
%\date {}
\begin{document}
\maketitle
%\tableofcontents


\section{Ejercicio 4}
\subsection{Apartado 1}
\textbf{Dado tu número n de 8 cifras de la lista del ejercicio 2.}

$$n = 77770081$$

\textbf{Factoriza n-1 aplicando el método $\rho$ de Polard. ¿Cúantas iteraciones necesitas?}

Se ha programado el método $\rho$ de Polard, y se ha aplicado recursivamente al número dado. Se ha tomado como $f(x)=x^2+1$ y como $x_0=1$.

Comenzamos aplicando el metodo $\rho$ de Polard a nuestro número $n-1$, que al ser par es claramente compuesto.

\begin{verbatim}
Paso: 0, x: 1, y: 1, g: -
Paso: 1, x: 2, y: 5, g: 3
\end{verbatim}

Como la descomposición es $77770080=3\cdot 25923360$, con 3 claramente primo y 25923360 es claramente compuesto al ser par. Para esto solo hemos necesitado 1 iteración.. Ahora aplicamos el método de nuevo a 25923360.

\begin{verbatim}
Paso: 0, x: 1, y: 1, g: -
Paso: 1, x: 2, y: 5, g: 3
\end{verbatim}

Como la descomposición es $25923360=3\cdot 8641120$ con 3 claramente primo y 8641120 es claramente compuesto al ser par. De nuevo, solo ha sido necesaria 1 iteración. Volvemos a aplicar el método de nuevo a 8641120.

\begin{verbatim}
Paso: 0, x: 1, y: 1, g: -
Paso: 1, x: 2, y: 5, g: 1
Paso: 2, x: 5, y: 677, g: 32
\end{verbatim}

Como la descomposición es $8641120=32\cdot 270035$, con 32 compuesto, pues es $32=2^5$. Dado que es una potencia de 2 y por tanto se descompone rápidamente a mano, no se le va a aplicar el algoritmo, y el cofactor 270035 es claramente compuesto al ser múltiplo de 5. En esta ocasión se han necesitado 2 iteraciones. Se le aplica el método de nuevo a 270035. 

\begin{verbatim}
Paso: 0, x: 1, y: 1, g: -
Paso: 1, x: 2, y: 5, g: 1
Paso: 2, x: 5, y: 677, g: 1
Paso: 3, x: 26, y: 221631, g: 5
\end{verbatim}

Como la descomposición es $270035=5\cdot 54007$, con 5 evidentemente primo y el cofactor es compuesto ya que $2^{54007-1}\equiv 29823\mod 54007\not\equiv 1\mod 54007$, de modo que aplicamos el algoritmo de nuevo a 54007.
En esta descomposición se han necesitado 3 iteraciones.

\begin{verbatim}
Paso: 0, x: 1, y: 1, g: -
Paso: 1, x: 2, y: 5, g: 1
Paso: 2, x: 5, y: 677, g: 1
Paso: 3, x: 26, y: 5603, g: 1
Paso: 4, x: 677, y: 11539, g: 1
Paso: 5, x: 26274, y: 30672, g: 1
Paso: 6, x: 5603, y: 12599, g: 53
\end{verbatim}

La descomposición de este número es $54007=53*1019$, para la que se han necesitado 6 iteraciones, donde mirando en la lista vemos que tanto 53 como 1019 son primos, luego la descomposición en factores de nuestro número es:

$n-1=3^2\cdot 2^5\cdot 5 \cdot 53 \cdot 1019$ y el total de iteraciones es $13$.



\subsection{Apartado 2}
\textbf{Si es necesario aplica recursivamente Lucas-Lehmer para certificar factores primos de n-1 mayores de 4 cifras.}

No ha resultado ningún factor mayor de 4 cifras, por lo que no es necesario.

\subsection{Apartado 3}

\textbf{Aplica Lucas-Lehmer para encontrar un certificado de primalidad de n.}

\textbf{NOTA: Debes encontrar el natural más pequeño cuya clase sea primitiva.}

Como ya conocemos los factores de {n-1} pues los hemos calculado en el primer apartado, $n-1=3^2\cdot 2^5\cdot 5 \cdot 53 \cdot 1019$, buscamos un elemento primitivo para $n$.

Vamos probando hasta encontrar que $a=17$ es un elemento primitivo para $n=77770081$ porque $17^{n-1}\equiv 1\mod n$ y $17^{\frac{n-1}{p}}\not\equiv 1\mod n $ para $p\in\{2,3,5,53,1019\}$ pues:

$$17^{\frac{n-1}{2}}\equiv 77770080\mod n$$
$$17^{\frac{n-1}{3}}\equiv 58134188\mod n$$
$$17^{\frac{n-1}{5}}\equiv 66432901\mod n$$
$$17^{\frac{n-1}{53}}\equiv 68065795\mod n$$
$$17^{\frac{n-1}{1019}}\equiv 65224721\mod n$$

luego por el Teorema de Lucas-Lehmer para $a=17$ tenemos que $n$ es primo.



\end{document}